\documentclass[a4paper]{article}
\usepackage[a4paper,left=3cm,right=2cm,top=2.5cm,bottom=2.5cm]{geometry}
\usepackage{palatino}
\usepackage{amsmath}
\usepackage[colorlinks=true,linkcolor=blue,citecolor=blue]{hyperref}
\usepackage{graphicx}
\usepackage{cp2021t}
\usepackage{subcaption}
\usepackage{adjustbox}
\usepackage{color}
\definecolor{red}{RGB}{255,  0,  0}
\definecolor{blue}{RGB}{0,0,255}
\def\red{\color{red}}
\def\blue{\color{blue}}
%================= local x=====================================================%
\def\getGif#1{\includegraphics[width=0.3\textwidth]{cp2021t_media/#1.png}}
\let\uk=\emph
\def\aspas#1{``#1"}
%================= lhs2tex=====================================================%
%% ODER: format ==         = "\mathrel{==}"
%% ODER: format /=         = "\neq "
%
%
\makeatletter
\@ifundefined{lhs2tex.lhs2tex.sty.read}%
  {\@namedef{lhs2tex.lhs2tex.sty.read}{}%
   \newcommand\SkipToFmtEnd{}%
   \newcommand\EndFmtInput{}%
   \long\def\SkipToFmtEnd#1\EndFmtInput{}%
  }\SkipToFmtEnd

\newcommand\ReadOnlyOnce[1]{\@ifundefined{#1}{\@namedef{#1}{}}\SkipToFmtEnd}
\usepackage{amstext}
\usepackage{amssymb}
\usepackage{stmaryrd}
\DeclareFontFamily{OT1}{cmtex}{}
\DeclareFontShape{OT1}{cmtex}{m}{n}
  {<5><6><7><8>cmtex8
   <9>cmtex9
   <10><10.95><12><14.4><17.28><20.74><24.88>cmtex10}{}
\DeclareFontShape{OT1}{cmtex}{m}{it}
  {<-> ssub * cmtt/m/it}{}
\newcommand{\texfamily}{\fontfamily{cmtex}\selectfont}
\DeclareFontShape{OT1}{cmtt}{bx}{n}
  {<5><6><7><8>cmtt8
   <9>cmbtt9
   <10><10.95><12><14.4><17.28><20.74><24.88>cmbtt10}{}
\DeclareFontShape{OT1}{cmtex}{bx}{n}
  {<-> ssub * cmtt/bx/n}{}
\newcommand{\tex}[1]{\text{\texfamily#1}}	% NEU

\newcommand{\Sp}{\hskip.33334em\relax}


\newcommand{\Conid}[1]{\mathit{#1}}
\newcommand{\Varid}[1]{\mathit{#1}}
\newcommand{\anonymous}{\kern0.06em \vbox{\hrule\@width.5em}}
\newcommand{\plus}{\mathbin{+\!\!\!+}}
\newcommand{\bind}{\mathbin{>\!\!\!>\mkern-6.7mu=}}
\newcommand{\rbind}{\mathbin{=\mkern-6.7mu<\!\!\!<}}% suggested by Neil Mitchell
\newcommand{\sequ}{\mathbin{>\!\!\!>}}
\renewcommand{\leq}{\leqslant}
\renewcommand{\geq}{\geqslant}
\usepackage{polytable}

%mathindent has to be defined
\@ifundefined{mathindent}%
  {\newdimen\mathindent\mathindent\leftmargini}%
  {}%

\def\resethooks{%
  \global\let\SaveRestoreHook\empty
  \global\let\ColumnHook\empty}
\newcommand*{\savecolumns}[1][default]%
  {\g@addto@macro\SaveRestoreHook{\savecolumns[#1]}}
\newcommand*{\restorecolumns}[1][default]%
  {\g@addto@macro\SaveRestoreHook{\restorecolumns[#1]}}
\newcommand*{\aligncolumn}[2]%
  {\g@addto@macro\ColumnHook{\column{#1}{#2}}}

\resethooks

\newcommand{\onelinecommentchars}{\quad-{}- }
\newcommand{\commentbeginchars}{\enskip\{-}
\newcommand{\commentendchars}{-\}\enskip}

\newcommand{\visiblecomments}{%
  \let\onelinecomment=\onelinecommentchars
  \let\commentbegin=\commentbeginchars
  \let\commentend=\commentendchars}

\newcommand{\invisiblecomments}{%
  \let\onelinecomment=\empty
  \let\commentbegin=\empty
  \let\commentend=\empty}

\visiblecomments

\newlength{\blanklineskip}
\setlength{\blanklineskip}{0.66084ex}

\newcommand{\hsindent}[1]{\quad}% default is fixed indentation
\let\hspre\empty
\let\hspost\empty
\newcommand{\NB}{\textbf{NB}}
\newcommand{\Todo}[1]{$\langle$\textbf{To do:}~#1$\rangle$}

\EndFmtInput
\makeatother
%
%
%
%
%
%
% This package provides two environments suitable to take the place
% of hscode, called "plainhscode" and "arrayhscode". 
%
% The plain environment surrounds each code block by vertical space,
% and it uses \abovedisplayskip and \belowdisplayskip to get spacing
% similar to formulas. Note that if these dimensions are changed,
% the spacing around displayed math formulas changes as well.
% All code is indented using \leftskip.
%
% Changed 19.08.2004 to reflect changes in colorcode. Should work with
% CodeGroup.sty.
%
\ReadOnlyOnce{polycode.fmt}%
\makeatletter

\newcommand{\hsnewpar}[1]%
  {{\parskip=0pt\parindent=0pt\par\vskip #1\noindent}}

% can be used, for instance, to redefine the code size, by setting the
% command to \small or something alike
\newcommand{\hscodestyle}{}

% The command \sethscode can be used to switch the code formatting
% behaviour by mapping the hscode environment in the subst directive
% to a new LaTeX environment.

\newcommand{\sethscode}[1]%
  {\expandafter\let\expandafter\hscode\csname #1\endcsname
   \expandafter\let\expandafter\endhscode\csname end#1\endcsname}

% "compatibility" mode restores the non-polycode.fmt layout.

\newenvironment{compathscode}%
  {\par\noindent
   \advance\leftskip\mathindent
   \hscodestyle
   \let\\=\@normalcr
   \let\hspre\(\let\hspost\)%
   \pboxed}%
  {\endpboxed\)%
   \par\noindent
   \ignorespacesafterend}

\newcommand{\compaths}{\sethscode{compathscode}}

% "plain" mode is the proposed default.
% It should now work with \centering.
% This required some changes. The old version
% is still available for reference as oldplainhscode.

\newenvironment{plainhscode}%
  {\hsnewpar\abovedisplayskip
   \advance\leftskip\mathindent
   \hscodestyle
   \let\hspre\(\let\hspost\)%
   \pboxed}%
  {\endpboxed%
   \hsnewpar\belowdisplayskip
   \ignorespacesafterend}

\newenvironment{oldplainhscode}%
  {\hsnewpar\abovedisplayskip
   \advance\leftskip\mathindent
   \hscodestyle
   \let\\=\@normalcr
   \(\pboxed}%
  {\endpboxed\)%
   \hsnewpar\belowdisplayskip
   \ignorespacesafterend}

% Here, we make plainhscode the default environment.

\newcommand{\plainhs}{\sethscode{plainhscode}}
\newcommand{\oldplainhs}{\sethscode{oldplainhscode}}
\plainhs

% The arrayhscode is like plain, but makes use of polytable's
% parray environment which disallows page breaks in code blocks.

\newenvironment{arrayhscode}%
  {\hsnewpar\abovedisplayskip
   \advance\leftskip\mathindent
   \hscodestyle
   \let\\=\@normalcr
   \(\parray}%
  {\endparray\)%
   \hsnewpar\belowdisplayskip
   \ignorespacesafterend}

\newcommand{\arrayhs}{\sethscode{arrayhscode}}

% The mathhscode environment also makes use of polytable's parray 
% environment. It is supposed to be used only inside math mode 
% (I used it to typeset the type rules in my thesis).

\newenvironment{mathhscode}%
  {\parray}{\endparray}

\newcommand{\mathhs}{\sethscode{mathhscode}}

% texths is similar to mathhs, but works in text mode.

\newenvironment{texthscode}%
  {\(\parray}{\endparray\)}

\newcommand{\texths}{\sethscode{texthscode}}

% The framed environment places code in a framed box.

\def\codeframewidth{\arrayrulewidth}
\RequirePackage{calc}

\newenvironment{framedhscode}%
  {\parskip=\abovedisplayskip\par\noindent
   \hscodestyle
   \arrayrulewidth=\codeframewidth
   \tabular{@{}|p{\linewidth-2\arraycolsep-2\arrayrulewidth-2pt}|@{}}%
   \hline\framedhslinecorrect\\{-1.5ex}%
   \let\endoflinesave=\\
   \let\\=\@normalcr
   \(\pboxed}%
  {\endpboxed\)%
   \framedhslinecorrect\endoflinesave{.5ex}\hline
   \endtabular
   \parskip=\belowdisplayskip\par\noindent
   \ignorespacesafterend}

\newcommand{\framedhslinecorrect}[2]%
  {#1[#2]}

\newcommand{\framedhs}{\sethscode{framedhscode}}

% The inlinehscode environment is an experimental environment
% that can be used to typeset displayed code inline.

\newenvironment{inlinehscode}%
  {\(\def\column##1##2{}%
   \let\>\undefined\let\<\undefined\let\\\undefined
   \newcommand\>[1][]{}\newcommand\<[1][]{}\newcommand\\[1][]{}%
   \def\fromto##1##2##3{##3}%
   \def\nextline{}}{\) }%

\newcommand{\inlinehs}{\sethscode{inlinehscode}}

% The joincode environment is a separate environment that
% can be used to surround and thereby connect multiple code
% blocks.

\newenvironment{joincode}%
  {\let\orighscode=\hscode
   \let\origendhscode=\endhscode
   \def\endhscode{\def\hscode{\endgroup\def\@currenvir{hscode}\\}\begingroup}
   %\let\SaveRestoreHook=\empty
   %\let\ColumnHook=\empty
   %\let\resethooks=\empty
   \orighscode\def\hscode{\endgroup\def\@currenvir{hscode}}}%
  {\origendhscode
   \global\let\hscode=\orighscode
   \global\let\endhscode=\origendhscode}%

\makeatother
\EndFmtInput
%
\def\ana#1{\mathopen{[\!(}#1\mathclose{)\!]}}
%%format (bin (n) (k)) = "\Big(\vcenter{\xymatrix@R=1pt{" n "\\" k "}}\Big)"


%---------------------------------------------------------------------------

\title{
       	Cálculo de Programas
\\
       	Trabalho Prático
\\
       	MiEI+LCC --- 2020/21
}

\author{
       	\dium
\\
       	Universidade do Minho
}


\date\mydate

\makeindex
\newcommand{\rn}[1]{\textcolor{red}{#1}}
\begin{document}

\maketitle

\begin{center}\large
\begin{tabular}{ll}
\textbf{Grupo} nr. & 74
\\\hline
a93269 & Inês Oliveira Anes Vicente
\\
a93308 & Jorge Miguel Silva Melo
\\
a93280 & Miguel Ângelo Machado Martins
\end{tabular}
\end{center}

\section{Preâmbulo}

\CP\ tem como objectivo principal ensinar
a progra\-mação de computadores como uma disciplina científica. Para isso
parte-se de um repertório de \emph{combinadores} que formam uma álgebra da
programação (conjunto de leis universais e seus corolários) e usam-se esses
combinadores para construir programas \emph{composicionalmente}, isto é,
agregando programas já existentes.

Na sequência pedagógica dos planos de estudo dos dois cursos que têm
esta disciplina, opta-se pela aplicação deste método à programação
em \Haskell\ (sem prejuízo da sua aplicação a outras linguagens
funcionais). Assim, o presente trabalho prático coloca os
alunos perante problemas concretos que deverão ser implementados em
\Haskell.  Há ainda um outro objectivo: o de ensinar a documentar
programas, a validá-los e a produzir textos técnico-científicos de
qualidade.

\section{Documentação} Para cumprir de forma integrada os objectivos
enunciados acima vamos recorrer a uma técnica de programa\-ção dita
``\litp{literária}'' \cite{Kn92}, cujo princípio base é o seguinte:
%
\begin{quote}\em Um programa e a sua documentação devem coincidir.
\end{quote}
%
Por outras palavras, o código fonte e a documentação de um
programa deverão estar no mesmo ficheiro.

O ficheiro \texttt{cp2021t.pdf} que está a ler é já um exemplo de
\litp{programação literária}: foi gerado a partir do texto fonte
\texttt{cp2021t.lhs}\footnote{O suffixo `lhs' quer dizer
\emph{\lhaskell{literate Haskell}}.} que encontrará no
\MaterialPedagogico\ desta disciplina descompactando o ficheiro
\texttt{cp2021t.zip} e executando:
\begin{Verbatim}[fontsize=\small]
    $ lhs2TeX cp2021t.lhs > cp2021t.tex
    $ pdflatex cp2021t
\end{Verbatim}
em que \href{https://hackage.haskell.org/package/lhs2tex}{\texttt\LhsToTeX} é
um pre-processador que faz ``pretty printing''
de código Haskell em \Latex\ e que deve desde já instalar executando
\begin{Verbatim}[fontsize=\small]
    $ cabal install lhs2tex --lib
\end{Verbatim}
Por outro lado, o mesmo ficheiro \texttt{cp2021t.lhs} é executável e contém
o ``kit'' básico, escrito em \Haskell, para realizar o trabalho. Basta executar
\begin{Verbatim}[fontsize=\small]
    $ ghci cp2021t.lhs
\end{Verbatim}


\noindent Abra o ficheiro \texttt{cp2021t.lhs} no seu editor de texto preferido
e verifique que assim é: todo o texto que se encontra dentro do ambiente
\begin{quote}\small\tt
\text{\ttfamily \char92{}begin\char123{}code\char125{}}
\\ ... \\
\text{\ttfamily \char92{}end\char123{}code\char125{}}
\end{quote}
é seleccionado pelo \GHCi\ para ser executado.

\section{Como realizar o trabalho}
Este trabalho teórico-prático deve ser realizado por grupos de 3 (ou 4) alunos.
Os detalhes da avaliação (datas para submissão do relatório e sua defesa
oral) são os que forem publicados na \cp{página da disciplina} na \emph{internet}.

Recomenda-se uma abordagem participativa dos membros do grupo
de trabalho por forma a poderem responder às questões que serão colocadas
na \emph{defesa oral} do relatório.

Em que consiste, então, o \emph{relatório} a que se refere o parágrafo anterior?
É a edição do texto que está a ser lido, preenchendo o anexo \ref{sec:resolucao}
com as respostas. O relatório deverá conter ainda a identificação dos membros
do grupo de trabalho, no local respectivo da folha de rosto.

Para gerar o PDF integral do relatório deve-se ainda correr os comando seguintes,
que actualizam a bibliografia (com \Bibtex) e o índice remissivo (com \Makeindex),
\begin{Verbatim}[fontsize=\small]
    $ bibtex cp2021t.aux
    $ makeindex cp2021t.idx
\end{Verbatim}
e recompilar o texto como acima se indicou. Dever-se-á ainda instalar o utilitário
\QuickCheck,
que ajuda a validar programas em \Haskell\ e a biblioteca \gloss{Gloss} para
geração de gráficos 2D:
\begin{Verbatim}[fontsize=\small]
    $ cabal install QuickCheck gloss --lib
\end{Verbatim}
Para testar uma propriedade \QuickCheck~\ensuremath{\Varid{prop}}, basta invocá-la com o comando:
\begin{tabbing}\ttfamily
~~~~~\char62{}~quickCheck~prop\\
\ttfamily ~~~~~\char43{}\char43{}\char43{}~OK\char44{}~passed~100~tests\char46{}
\end{tabbing}
Pode-se ainda controlar o número de casos de teste e sua complexidade,
como o seguinte exemplo mostra:
\begin{tabbing}\ttfamily
~~~~~\char62{}~quickCheckWith~stdArgs~\char123{}~maxSuccess~\char61{}~200\char44{}~maxSize~\char61{}~10~\char125{}~prop\\
\ttfamily ~~~~~\char43{}\char43{}\char43{}~OK\char44{}~passed~200~tests\char46{}
\end{tabbing}
Qualquer programador tem, na vida real, de ler e analisar (muito!) código
escrito por outros. No anexo \ref{sec:codigo} disponibiliza-se algum
código \Haskell\ relativo aos problemas que se seguem. Esse anexo deverá
ser consultado e analisado à medida que isso for necessário.

\subsection{Stack}

O \stack{Stack} é um programa útil para criar, gerir e manter projetos em \Haskell.
Um projeto criado com o Stack possui uma estrutura de pastas muito específica:

\begin{itemize}
\item Os módulos auxiliares encontram-se na pasta \emph{src}.
\item O módulos principal encontra-se na pasta \emph{app}.
\item A lista de depêndencias externas encontra-se no ficheiro \emph{package.yaml}.
\end{itemize}

Pode aceder ao \GHCi\ utilizando o comando:
\begin{tabbing}\ttfamily
~stack~ghci
\end{tabbing}

Garanta que se encontra na pasta mais externa \textbf{do projeto}.
A primeira vez que correr este comando as depêndencias externas serão instaladas automaticamente.

Para gerar o PDF, garanta que se encontra na diretoria \emph{app}.

\Problema

Os \emph{tipos de dados algébricos} estudados ao longo desta disciplina oferecem
uma grande capacidade expressiva ao programador. Graças à sua flexibilidade,
torna-se trivial implementar \DSL s
e até mesmo \href{http://www.cse.chalmers.se/~ulfn/papers/thesis.pdf}{linguagens de programação}.

Paralelamente, um tópico bastante estudado no âmbito de \DL\
é a derivação automática de expressões matemáticas, por exemplo, de derivadas.
Duas técnicas que podem ser utilizadas para o cálculo de derivadas são:

\begin{itemize}
\item \emph{Symbolic differentiation}
\item \emph{Automatic differentiation}
\end{itemize}

\emph{Symbolic differentiation} consiste na aplicação sucessiva de transformações
(leia-se: funções) que sejam congruentes com as regras de derivação. O resultado
final será a expressão da derivada.

O leitor atento poderá notar um problema desta técnica: a expressão
inicial pode crescer de forma descontrolada, levando a um cálculo pouco eficiente.
\emph{Automatic differentiation} tenta resolver este problema,
calculando \textbf{o valor} da derivada da expressão em todos os passos.
Para tal, é necessário calcular o valor da expressão \textbf{e} o valor da sua derivada.

Vamos de seguida definir uma linguagem de expressões matemáticas simples e
implementar as duas técnicas de derivação automática.
Para isso, seja dado o seguinte tipo de dados,

\begin{hscode}\SaveRestoreHook
\column{B}{@{}>{\hspre}l<{\hspost}@{}}%
\column{12}{@{}>{\hspre}l<{\hspost}@{}}%
\column{E}{@{}>{\hspre}l<{\hspost}@{}}%
\>[B]{}\mathbf{data}\;\Conid{ExpAr}\;\Varid{a}\mathrel{=}\Conid{X}{}\<[E]%
\\
\>[B]{}\hsindent{12}{}\<[12]%
\>[12]{}\mid \Conid{N}\;\Varid{a}{}\<[E]%
\\
\>[B]{}\hsindent{12}{}\<[12]%
\>[12]{}\mid \Conid{Bin}\;\Conid{BinOp}\;(\Conid{ExpAr}\;\Varid{a})\;(\Conid{ExpAr}\;\Varid{a}){}\<[E]%
\\
\>[B]{}\hsindent{12}{}\<[12]%
\>[12]{}\mid \Conid{Un}\;\Conid{UnOp}\;(\Conid{ExpAr}\;\Varid{a}){}\<[E]%
\\
\>[B]{}\hsindent{12}{}\<[12]%
\>[12]{}\mathbf{deriving}\;(\Conid{Eq},\Conid{Show}){}\<[E]%
\ColumnHook
\end{hscode}\resethooks

\noindent
onde \ensuremath{\Conid{BinOp}} e \ensuremath{\Conid{UnOp}} representam operações binárias e unárias, respectivamente:

\begin{hscode}\SaveRestoreHook
\column{B}{@{}>{\hspre}l<{\hspost}@{}}%
\column{11}{@{}>{\hspre}l<{\hspost}@{}}%
\column{12}{@{}>{\hspre}l<{\hspost}@{}}%
\column{E}{@{}>{\hspre}l<{\hspost}@{}}%
\>[B]{}\mathbf{data}\;\Conid{BinOp}\mathrel{=}\Conid{Sum}{}\<[E]%
\\
\>[B]{}\hsindent{12}{}\<[12]%
\>[12]{}\mid \Conid{Product}{}\<[E]%
\\
\>[B]{}\hsindent{12}{}\<[12]%
\>[12]{}\mathbf{deriving}\;(\Conid{Eq},\Conid{Show}){}\<[E]%
\\[\blanklineskip]%
\>[B]{}\mathbf{data}\;\Conid{UnOp}\mathrel{=}\Conid{Negate}{}\<[E]%
\\
\>[B]{}\hsindent{11}{}\<[11]%
\>[11]{}\mid \Conid{E}{}\<[E]%
\\
\>[B]{}\hsindent{11}{}\<[11]%
\>[11]{}\mathbf{deriving}\;(\Conid{Eq},\Conid{Show}){}\<[E]%
\ColumnHook
\end{hscode}\resethooks

\noindent
O construtor \ensuremath{\Conid{E}} simboliza o exponencial de base $e$.

Assim, cada expressão pode ser uma variável, um número, uma operação binária
aplicada às devidas expressões, ou uma operação unária aplicada a uma expressão.
Por exemplo,
\begin{hscode}\SaveRestoreHook
\column{B}{@{}>{\hspre}l<{\hspost}@{}}%
\column{E}{@{}>{\hspre}l<{\hspost}@{}}%
\>[B]{}\Conid{Bin}\;\Conid{Sum}\;\Conid{X}\;(\Conid{N}\;\mathrm{10}){}\<[E]%
\ColumnHook
\end{hscode}\resethooks
designa \ensuremath{\Varid{x}\mathbin{+}\mathrm{10}} na notação matemática habitual.

\begin{enumerate}
\item A definição das funções \ensuremath{\Varid{inExpAr}} e \ensuremath{\Varid{baseExpAr}} para este tipo é a seguinte:
\begin{hscode}\SaveRestoreHook
\column{B}{@{}>{\hspre}l<{\hspost}@{}}%
\column{3}{@{}>{\hspre}l<{\hspost}@{}}%
\column{11}{@{}>{\hspre}l<{\hspost}@{}}%
\column{E}{@{}>{\hspre}l<{\hspost}@{}}%
\>[B]{}\Varid{inExpAr}\mathrel{=}\alt{\underline{\Conid{X}}}{\Varid{num\char95 ops}}\;\mathbf{where}{}\<[E]%
\\
\>[B]{}\hsindent{3}{}\<[3]%
\>[3]{}\Varid{num\char95 ops}\mathrel{=}\alt{\Conid{N}}{\Varid{ops}}{}\<[E]%
\\
\>[B]{}\hsindent{3}{}\<[3]%
\>[3]{}\Varid{ops}{}\<[11]%
\>[11]{}\mathrel{=}\alt{\Varid{bin}}{\uncurry{\Conid{Un}}}{}\<[E]%
\\
\>[B]{}\hsindent{3}{}\<[3]%
\>[3]{}\Varid{bin}\;(\Varid{op},(\Varid{a},\Varid{b}))\mathrel{=}\Conid{Bin}\;\Varid{op}\;\Varid{a}\;\Varid{b}{}\<[E]%
\\[\blanklineskip]%
\>[B]{}\Varid{baseExpAr}\;\Varid{f}\;\Varid{g}\;\Varid{h}\;\Varid{j}\;\Varid{k}\;\Varid{l}\;\Varid{z}\mathrel{=}\Varid{f}+(\Varid{g}+(\Varid{h}\times(\Varid{j}\times\Varid{k})+\Varid{l}\times\Varid{z})){}\<[E]%
\ColumnHook
\end{hscode}\resethooks

  Defina as funções \ensuremath{\Varid{outExpAr}} e \ensuremath{\Varid{recExpAr}},
  e teste as propriedades que se seguem.
  \begin{propriedade}
    \ensuremath{\Varid{inExpAr}} e \ensuremath{\Varid{outExpAr}} são testemunhas de um isomorfismo,
    isto é,
    \ensuremath{\Varid{inExpAr}\comp \Varid{outExpAr}\mathrel{=}\Varid{id}} e \ensuremath{\Varid{outExpAr}\comp \Varid{idExpAr}\mathrel{=}\Varid{id}}:
\begin{hscode}\SaveRestoreHook
\column{B}{@{}>{\hspre}l<{\hspost}@{}}%
\column{E}{@{}>{\hspre}l<{\hspost}@{}}%
\>[B]{}\Varid{prop\char95 in\char95 out\char95 idExpAr}\mathbin{::}(\Conid{Eq}\;\Varid{a})\Rightarrow \Conid{ExpAr}\;\Varid{a}\to \Conid{Bool}{}\<[E]%
\\
\>[B]{}\Varid{prop\char95 in\char95 out\char95 idExpAr}\mathrel{=}\Varid{inExpAr}\comp \Varid{outExpAr}\equiv\Varid{id}{}\<[E]%
\\[\blanklineskip]%
\>[B]{}\Varid{prop\char95 out\char95 in\char95 idExpAr}\mathbin{::}(\Conid{Eq}\;\Varid{a})\Rightarrow \Conid{OutExpAr}\;\Varid{a}\to \Conid{Bool}{}\<[E]%
\\
\>[B]{}\Varid{prop\char95 out\char95 in\char95 idExpAr}\mathrel{=}\Varid{outExpAr}\comp \Varid{inExpAr}\equiv\Varid{id}{}\<[E]%
\ColumnHook
\end{hscode}\resethooks
    \end{propriedade}

  \item Dada uma expressão aritmética e um escalar para substituir o \ensuremath{\Conid{X}},
	a função

\begin{quote}
      \ensuremath{\Varid{eval\char95 exp}\mathbin{::}\Conid{Floating}\;\Varid{a}\Rightarrow \Varid{a}\to (\Conid{ExpAr}\;\Varid{a})\to \Varid{a}}
\end{quote}

\noindent calcula o resultado da expressão. Na página \pageref{pg:P1}
    esta função está expressa como um catamorfismo. Defina o respectivo gene
    e, de seguida, teste as propriedades:
    \begin{propriedade}
       A função \ensuremath{\Varid{eval\char95 exp}} respeita os elementos neutros das operações.
\begin{hscode}\SaveRestoreHook
\column{B}{@{}>{\hspre}l<{\hspost}@{}}%
\column{3}{@{}>{\hspre}l<{\hspost}@{}}%
\column{E}{@{}>{\hspre}l<{\hspost}@{}}%
\>[B]{}\Varid{prop\char95 sum\char95 idr}\mathbin{::}(\Conid{Floating}\;\Varid{a},\Conid{Real}\;\Varid{a})\Rightarrow \Varid{a}\to \Conid{ExpAr}\;\Varid{a}\to \Conid{Bool}{}\<[E]%
\\
\>[B]{}\Varid{prop\char95 sum\char95 idr}\;\Varid{a}\;\Varid{exp}\mathrel{=}\Varid{eval\char95 exp}\;\Varid{a}\;\Varid{exp}\mathbin{\stackrel{\mathrm{?}}{=}}\Varid{sum\char95 idr}\;\mathbf{where}{}\<[E]%
\\
\>[B]{}\hsindent{3}{}\<[3]%
\>[3]{}\Varid{sum\char95 idr}\mathrel{=}\Varid{eval\char95 exp}\;\Varid{a}\;(\Conid{Bin}\;\Conid{Sum}\;\Varid{exp}\;(\Conid{N}\;\mathrm{0})){}\<[E]%
\\[\blanklineskip]%
\>[B]{}\Varid{prop\char95 sum\char95 idl}\mathbin{::}(\Conid{Floating}\;\Varid{a},\Conid{Real}\;\Varid{a})\Rightarrow \Varid{a}\to \Conid{ExpAr}\;\Varid{a}\to \Conid{Bool}{}\<[E]%
\\
\>[B]{}\Varid{prop\char95 sum\char95 idl}\;\Varid{a}\;\Varid{exp}\mathrel{=}\Varid{eval\char95 exp}\;\Varid{a}\;\Varid{exp}\mathbin{\stackrel{\mathrm{?}}{=}}\Varid{sum\char95 idl}\;\mathbf{where}{}\<[E]%
\\
\>[B]{}\hsindent{3}{}\<[3]%
\>[3]{}\Varid{sum\char95 idl}\mathrel{=}\Varid{eval\char95 exp}\;\Varid{a}\;(\Conid{Bin}\;\Conid{Sum}\;(\Conid{N}\;\mathrm{0})\;\Varid{exp}){}\<[E]%
\\[\blanklineskip]%
\>[B]{}\Varid{prop\char95 product\char95 idr}\mathbin{::}(\Conid{Floating}\;\Varid{a},\Conid{Real}\;\Varid{a})\Rightarrow \Varid{a}\to \Conid{ExpAr}\;\Varid{a}\to \Conid{Bool}{}\<[E]%
\\
\>[B]{}\Varid{prop\char95 product\char95 idr}\;\Varid{a}\;\Varid{exp}\mathrel{=}\Varid{eval\char95 exp}\;\Varid{a}\;\Varid{exp}\mathbin{\stackrel{\mathrm{?}}{=}}\Varid{prod\char95 idr}\;\mathbf{where}{}\<[E]%
\\
\>[B]{}\hsindent{3}{}\<[3]%
\>[3]{}\Varid{prod\char95 idr}\mathrel{=}\Varid{eval\char95 exp}\;\Varid{a}\;(\Conid{Bin}\;\Conid{Product}\;\Varid{exp}\;(\Conid{N}\;\mathrm{1})){}\<[E]%
\\[\blanklineskip]%
\>[B]{}\Varid{prop\char95 product\char95 idl}\mathbin{::}(\Conid{Floating}\;\Varid{a},\Conid{Real}\;\Varid{a})\Rightarrow \Varid{a}\to \Conid{ExpAr}\;\Varid{a}\to \Conid{Bool}{}\<[E]%
\\
\>[B]{}\Varid{prop\char95 product\char95 idl}\;\Varid{a}\;\Varid{exp}\mathrel{=}\Varid{eval\char95 exp}\;\Varid{a}\;\Varid{exp}\mathbin{\stackrel{\mathrm{?}}{=}}\Varid{prod\char95 idl}\;\mathbf{where}{}\<[E]%
\\
\>[B]{}\hsindent{3}{}\<[3]%
\>[3]{}\Varid{prod\char95 idl}\mathrel{=}\Varid{eval\char95 exp}\;\Varid{a}\;(\Conid{Bin}\;\Conid{Product}\;(\Conid{N}\;\mathrm{1})\;\Varid{exp}){}\<[E]%
\\[\blanklineskip]%
\>[B]{}\Varid{prop\char95 e\char95 id}\mathbin{::}(\Conid{Floating}\;\Varid{a},\Conid{Real}\;\Varid{a})\Rightarrow \Varid{a}\to \Conid{Bool}{}\<[E]%
\\
\>[B]{}\Varid{prop\char95 e\char95 id}\;\Varid{a}\mathrel{=}\Varid{eval\char95 exp}\;\Varid{a}\;(\Conid{Un}\;\Conid{E}\;(\Conid{N}\;\mathrm{1}))\equiv \Varid{expd}\;\mathrm{1}{}\<[E]%
\\[\blanklineskip]%
\>[B]{}\Varid{prop\char95 negate\char95 id}\mathbin{::}(\Conid{Floating}\;\Varid{a},\Conid{Real}\;\Varid{a})\Rightarrow \Varid{a}\to \Conid{Bool}{}\<[E]%
\\
\>[B]{}\Varid{prop\char95 negate\char95 id}\;\Varid{a}\mathrel{=}\Varid{eval\char95 exp}\;\Varid{a}\;(\Conid{Un}\;\Conid{Negate}\;(\Conid{N}\;\mathrm{0}))\equiv \mathrm{0}{}\<[E]%
\ColumnHook
\end{hscode}\resethooks
    \end{propriedade}
    \begin{propriedade}
      Negar duas vezes uma expressão tem o mesmo valor que não fazer nada.
\begin{hscode}\SaveRestoreHook
\column{B}{@{}>{\hspre}l<{\hspost}@{}}%
\column{E}{@{}>{\hspre}l<{\hspost}@{}}%
\>[B]{}\Varid{prop\char95 double\char95 negate}\mathbin{::}(\Conid{Floating}\;\Varid{a},\Conid{Real}\;\Varid{a})\Rightarrow \Varid{a}\to \Conid{ExpAr}\;\Varid{a}\to \Conid{Bool}{}\<[E]%
\\
\>[B]{}\Varid{prop\char95 double\char95 negate}\;\Varid{a}\;\Varid{exp}\mathrel{=}\Varid{eval\char95 exp}\;\Varid{a}\;\Varid{exp}\mathbin{\stackrel{\mathrm{?}}{=}}\Varid{eval\char95 exp}\;\Varid{a}\;(\Conid{Un}\;\Conid{Negate}\;(\Conid{Un}\;\Conid{Negate}\;\Varid{exp})){}\<[E]%
\ColumnHook
\end{hscode}\resethooks
    \end{propriedade}

  \item É possível otimizar o cálculo do valor de uma expressão aritmética tirando proveito
  dos elementos absorventes de cada operação. Implemente os genes da função
\begin{hscode}\SaveRestoreHook
\column{B}{@{}>{\hspre}l<{\hspost}@{}}%
\column{7}{@{}>{\hspre}l<{\hspost}@{}}%
\column{E}{@{}>{\hspre}l<{\hspost}@{}}%
\>[7]{}\Varid{optmize\char95 eval}\mathbin{::}(\Conid{Floating}\;\Varid{a},\Conid{Eq}\;\Varid{a})\Rightarrow \Varid{a}\to (\Conid{ExpAr}\;\Varid{a})\to \Varid{a}{}\<[E]%
\ColumnHook
\end{hscode}\resethooks
  que se encontra na página \pageref{pg:P1} expressa como um hilomorfismo\footnote{Qual é a vantagem de implementar a função \ensuremath{\Varid{optimize\char95 eval}} utilizando um hilomorfismo em vez de utilizar um catamorfismo com um gene "inteligente"?}
  e teste as propriedades:

    \begin{propriedade}
      A função \ensuremath{\Varid{optimize\char95 eval}} respeita a semântica da função \ensuremath{\Varid{eval}}.
\begin{hscode}\SaveRestoreHook
\column{B}{@{}>{\hspre}l<{\hspost}@{}}%
\column{E}{@{}>{\hspre}l<{\hspost}@{}}%
\>[B]{}\Varid{prop\char95 optimize\char95 respects\char95 semantics}\mathbin{::}(\Conid{Floating}\;\Varid{a},\Conid{Real}\;\Varid{a})\Rightarrow \Varid{a}\to \Conid{ExpAr}\;\Varid{a}\to \Conid{Bool}{}\<[E]%
\\
\>[B]{}\Varid{prop\char95 optimize\char95 respects\char95 semantics}\;\Varid{a}\;\Varid{exp}\mathrel{=}\Varid{eval\char95 exp}\;\Varid{a}\;\Varid{exp}\mathbin{\stackrel{\mathrm{?}}{=}}\Varid{optmize\char95 eval}\;\Varid{a}\;\Varid{exp}{}\<[E]%
\ColumnHook
\end{hscode}\resethooks
    \end{propriedade}


\item Para calcular a derivada de uma expressão, é necessário aplicar transformações
à expressão original que respeitem as regras das derivadas:\footnote{%
	Apesar da adição e multiplicação gozarem da propriedade comutativa,
	há que ter em atenção a ordem das operações por causa dos testes.}

\begin{itemize}
  \item Regra da soma:
\begin{eqnarray*}
	\frac{d}{dx}(f(x)+g(x))=\frac{d}{dx}(f(x))+\frac{d}{dx}(g(x))
\end{eqnarray*}
  \item Regra do produto:
\begin{eqnarray*}
	\frac{d}{dx}(f(x)g(x))=f(x)\cdot \frac{d}{dx}(g(x))+\frac{d}{dx}(f(x))\cdot g(x)
\end{eqnarray*}
\end{itemize}

  Defina o gene do catamorfismo que ocorre na função
    \begin{quote}
      \ensuremath{\Varid{sd}\mathbin{::}\Conid{Floating}\;\Varid{a}\Rightarrow \Conid{ExpAr}\;\Varid{a}\to \Conid{ExpAr}\;\Varid{a}}
    \end{quote}
  que, dada uma expressão aritmética, calcula a sua derivada.
  Testes a fazer, de seguida:
    \begin{propriedade}
       A função \ensuremath{\Varid{sd}} respeita as regras de derivação.
\begin{hscode}\SaveRestoreHook
\column{B}{@{}>{\hspre}l<{\hspost}@{}}%
\column{3}{@{}>{\hspre}l<{\hspost}@{}}%
\column{E}{@{}>{\hspre}l<{\hspost}@{}}%
\>[B]{}\Varid{prop\char95 const\char95 rule}\mathbin{::}(\Conid{Real}\;\Varid{a},\Conid{Floating}\;\Varid{a})\Rightarrow \Varid{a}\to \Conid{Bool}{}\<[E]%
\\
\>[B]{}\Varid{prop\char95 const\char95 rule}\;\Varid{a}\mathrel{=}\Varid{sd}\;(\Conid{N}\;\Varid{a})\equiv \Conid{N}\;\mathrm{0}{}\<[E]%
\\[\blanklineskip]%
\>[B]{}\Varid{prop\char95 var\char95 rule}\mathbin{::}\Conid{Bool}{}\<[E]%
\\
\>[B]{}\Varid{prop\char95 var\char95 rule}\mathrel{=}\Varid{sd}\;\Conid{X}\equiv \Conid{N}\;\mathrm{1}{}\<[E]%
\\[\blanklineskip]%
\>[B]{}\Varid{prop\char95 sum\char95 rule}\mathbin{::}(\Conid{Real}\;\Varid{a},\Conid{Floating}\;\Varid{a})\Rightarrow \Conid{ExpAr}\;\Varid{a}\to \Conid{ExpAr}\;\Varid{a}\to \Conid{Bool}{}\<[E]%
\\
\>[B]{}\Varid{prop\char95 sum\char95 rule}\;\Varid{exp1}\;\Varid{exp2}\mathrel{=}\Varid{sd}\;(\Conid{Bin}\;\Conid{Sum}\;\Varid{exp1}\;\Varid{exp2})\equiv \Varid{sum\char95 rule}\;\mathbf{where}{}\<[E]%
\\
\>[B]{}\hsindent{3}{}\<[3]%
\>[3]{}\Varid{sum\char95 rule}\mathrel{=}\Conid{Bin}\;\Conid{Sum}\;(\Varid{sd}\;\Varid{exp1})\;(\Varid{sd}\;\Varid{exp2}){}\<[E]%
\\[\blanklineskip]%
\>[B]{}\Varid{prop\char95 product\char95 rule}\mathbin{::}(\Conid{Real}\;\Varid{a},\Conid{Floating}\;\Varid{a})\Rightarrow \Conid{ExpAr}\;\Varid{a}\to \Conid{ExpAr}\;\Varid{a}\to \Conid{Bool}{}\<[E]%
\\
\>[B]{}\Varid{prop\char95 product\char95 rule}\;\Varid{exp1}\;\Varid{exp2}\mathrel{=}\Varid{sd}\;(\Conid{Bin}\;\Conid{Product}\;\Varid{exp1}\;\Varid{exp2})\equiv \Varid{prod\char95 rule}\;\mathbf{where}{}\<[E]%
\\
\>[B]{}\hsindent{3}{}\<[3]%
\>[3]{}\Varid{prod\char95 rule}\mathrel{=}\Conid{Bin}\;\Conid{Sum}\;(\Conid{Bin}\;\Conid{Product}\;\Varid{exp1}\;(\Varid{sd}\;\Varid{exp2}))\;(\Conid{Bin}\;\Conid{Product}\;(\Varid{sd}\;\Varid{exp1})\;\Varid{exp2}){}\<[E]%
\\[\blanklineskip]%
\>[B]{}\Varid{prop\char95 e\char95 rule}\mathbin{::}(\Conid{Real}\;\Varid{a},\Conid{Floating}\;\Varid{a})\Rightarrow \Conid{ExpAr}\;\Varid{a}\to \Conid{Bool}{}\<[E]%
\\
\>[B]{}\Varid{prop\char95 e\char95 rule}\;\Varid{exp}\mathrel{=}\Varid{sd}\;(\Conid{Un}\;\Conid{E}\;\Varid{exp})\equiv \Conid{Bin}\;\Conid{Product}\;(\Conid{Un}\;\Conid{E}\;\Varid{exp})\;(\Varid{sd}\;\Varid{exp}){}\<[E]%
\\[\blanklineskip]%
\>[B]{}\Varid{prop\char95 negate\char95 rule}\mathbin{::}(\Conid{Real}\;\Varid{a},\Conid{Floating}\;\Varid{a})\Rightarrow \Conid{ExpAr}\;\Varid{a}\to \Conid{Bool}{}\<[E]%
\\
\>[B]{}\Varid{prop\char95 negate\char95 rule}\;\Varid{exp}\mathrel{=}\Varid{sd}\;(\Conid{Un}\;\Conid{Negate}\;\Varid{exp})\equiv \Conid{Un}\;\Conid{Negate}\;(\Varid{sd}\;\Varid{exp}){}\<[E]%
\ColumnHook
\end{hscode}\resethooks
    \end{propriedade}

\item Como foi visto, \emph{Symbolic differentiation} não é a técnica
mais eficaz para o cálculo do valor da derivada de uma expressão.
\emph{Automatic differentiation} resolve este problema cálculando o valor
da derivada em vez de manipular a expressão original.

  Defina o gene do catamorfismo que ocorre na função
    \begin{hscode}\SaveRestoreHook
\column{B}{@{}>{\hspre}l<{\hspost}@{}}%
\column{5}{@{}>{\hspre}l<{\hspost}@{}}%
\column{E}{@{}>{\hspre}l<{\hspost}@{}}%
\>[5]{}\Varid{ad}\mathbin{::}\Conid{Floating}\;\Varid{a}\Rightarrow \Varid{a}\to \Conid{ExpAr}\;\Varid{a}\to \Varid{a}{}\<[E]%
\ColumnHook
\end{hscode}\resethooks
  que, dada uma expressão aritmética e um ponto,
  calcula o valor da sua derivada nesse ponto,
  sem transformar manipular a expressão original.
  Testes a fazer, de seguida:

    \begin{propriedade}
       Calcular o valor da derivada num ponto \ensuremath{\Varid{r}} via \ensuremath{\Varid{ad}} é equivalente a calcular a derivada da expressão e avalia-la no ponto \ensuremath{\Varid{r}}.
\begin{hscode}\SaveRestoreHook
\column{B}{@{}>{\hspre}l<{\hspost}@{}}%
\column{E}{@{}>{\hspre}l<{\hspost}@{}}%
\>[B]{}\Varid{prop\char95 congruent}\mathbin{::}(\Conid{Floating}\;\Varid{a},\Conid{Real}\;\Varid{a})\Rightarrow \Varid{a}\to \Conid{ExpAr}\;\Varid{a}\to \Conid{Bool}{}\<[E]%
\\
\>[B]{}\Varid{prop\char95 congruent}\;\Varid{a}\;\Varid{exp}\mathrel{=}\Varid{ad}\;\Varid{a}\;\Varid{exp}\mathbin{\stackrel{\mathrm{?}}{=}}\Varid{eval\char95 exp}\;\Varid{a}\;(\Varid{sd}\;\Varid{exp}){}\<[E]%
\ColumnHook
\end{hscode}\resethooks
    \end{propriedade}
\end{enumerate}

\Problema

Nesta disciplina estudou-se como fazer \pd{programação dinâmica} por cálculo,
recorrendo à lei de recursividade mútua.\footnote{Lei (\ref{eq:fokkinga})
em \cite{Ol18}, página \pageref{eq:fokkinga}.}

Para o caso de funções sobre os números naturais (\ensuremath{\N_0}, com functor \ensuremath{\fun F \;\Conid{X}\mathrel{=}\mathrm{1}\mathbin{+}\Conid{X}}) é fácil derivar-se da lei que foi estudada uma
	\emph{regra de algibeira}
	\label{pg:regra}
que se pode ensinar a programadores que não tenham estudado
\cp{Cálculo de Programas}. Apresenta-se de seguida essa regra, tomando como exemplo o
cálculo do ciclo-\textsf{for} que implementa a função de Fibonacci, recordar
o sistema
\begin{hscode}\SaveRestoreHook
\column{B}{@{}>{\hspre}l<{\hspost}@{}}%
\column{E}{@{}>{\hspre}l<{\hspost}@{}}%
\>[B]{}\Varid{fib}\;\mathrm{0}\mathrel{=}\mathrm{1}{}\<[E]%
\\
\>[B]{}\Varid{fib}\;(\Varid{n}\mathbin{+}\mathrm{1})\mathrel{=}\Varid{f}\;\Varid{n}{}\<[E]%
\\[\blanklineskip]%
\>[B]{}\Varid{f}\;\mathrm{0}\mathrel{=}\mathrm{1}{}\<[E]%
\\
\>[B]{}\Varid{f}\;(\Varid{n}\mathbin{+}\mathrm{1})\mathrel{=}\Varid{fib}\;\Varid{n}\mathbin{+}\Varid{f}\;\Varid{n}{}\<[E]%
\ColumnHook
\end{hscode}\resethooks
Obter-se-á de imediato
\begin{hscode}\SaveRestoreHook
\column{B}{@{}>{\hspre}l<{\hspost}@{}}%
\column{4}{@{}>{\hspre}l<{\hspost}@{}}%
\column{E}{@{}>{\hspre}l<{\hspost}@{}}%
\>[B]{}\Varid{fib'}\mathrel{=}\p1\comp \for{\Varid{loop}}\ {\Varid{init}}\;\mathbf{where}{}\<[E]%
\\
\>[B]{}\hsindent{4}{}\<[4]%
\>[4]{}\Varid{loop}\;(\Varid{fib},\Varid{f})\mathrel{=}(\Varid{f},\Varid{fib}\mathbin{+}\Varid{f}){}\<[E]%
\\
\>[B]{}\hsindent{4}{}\<[4]%
\>[4]{}\Varid{init}\mathrel{=}(\mathrm{1},\mathrm{1}){}\<[E]%
\ColumnHook
\end{hscode}\resethooks
usando as regras seguintes:
\begin{itemize}
\item	O corpo do ciclo \ensuremath{\Varid{loop}} terá tantos argumentos quanto o número de funções mutuamente recursivas.
\item	Para as variáveis escolhem-se os próprios nomes das funções, pela ordem
que se achar conveniente.\footnote{Podem obviamente usar-se outros símbolos, mas numa primeira leitura
dá jeito usarem-se tais nomes.}
\item	Para os resultados vão-se buscar as expressões respectivas, retirando a variável \ensuremath{\Varid{n}}.
\item	Em \ensuremath{\Varid{init}} coleccionam-se os resultados dos casos de base das funções, pela mesma ordem.
\end{itemize}
Mais um exemplo, envolvendo polinómios do segundo grau $ax^2 + b x + c$ em \ensuremath{\N_0}.
Seguindo o método estudado nas aulas\footnote{Secção 3.17 de \cite{Ol18} e tópico
\href{https://www4.di.uminho.pt/~jno/media/cp/}{Recursividade mútua} nos vídeos das aulas teóricas.},
de $f\ x = a x^2 + b x + c$ derivam-se duas funções mutuamente recursivas:
\begin{hscode}\SaveRestoreHook
\column{B}{@{}>{\hspre}l<{\hspost}@{}}%
\column{E}{@{}>{\hspre}l<{\hspost}@{}}%
\>[B]{}\Varid{f}\;\mathrm{0}\mathrel{=}\Varid{c}{}\<[E]%
\\
\>[B]{}\Varid{f}\;(\Varid{n}\mathbin{+}\mathrm{1})\mathrel{=}\Varid{f}\;\Varid{n}\mathbin{+}\Varid{k}\;\Varid{n}{}\<[E]%
\\[\blanklineskip]%
\>[B]{}\Varid{k}\;\mathrm{0}\mathrel{=}\Varid{a}\mathbin{+}\Varid{b}{}\<[E]%
\\
\>[B]{}\Varid{k}\;(\Varid{n}\mathbin{+}\mathrm{1})\mathrel{=}\Varid{k}\;\Varid{n}\mathbin{+}\mathrm{2}\;\Varid{a}{}\<[E]%
\ColumnHook
\end{hscode}\resethooks
Seguindo a regra acima, calcula-se de imediato a seguinte implementação, em Haskell:
\begin{hscode}\SaveRestoreHook
\column{B}{@{}>{\hspre}l<{\hspost}@{}}%
\column{3}{@{}>{\hspre}l<{\hspost}@{}}%
\column{E}{@{}>{\hspre}l<{\hspost}@{}}%
\>[B]{}\Varid{f'}\;\Varid{a}\;\Varid{b}\;\Varid{c}\mathrel{=}\p1\comp \for{\Varid{loop}}\ {\Varid{init}}\;\mathbf{where}{}\<[E]%
\\
\>[B]{}\hsindent{3}{}\<[3]%
\>[3]{}\Varid{loop}\;(\Varid{f},\Varid{k})\mathrel{=}(\Varid{f}\mathbin{+}\Varid{k},\Varid{k}\mathbin{+}\mathrm{2}\mathbin{*}\Varid{a}){}\<[E]%
\\
\>[B]{}\hsindent{3}{}\<[3]%
\>[3]{}\Varid{init}\mathrel{=}(\Varid{c},\Varid{a}\mathbin{+}\Varid{b}){}\<[E]%
\ColumnHook
\end{hscode}\resethooks
O que se pede então, nesta pergunta?
Dada a fórmula que dá o \ensuremath{\Varid{n}}-ésimo \catalan{número de Catalan},
\begin{eqnarray}
	C_n = \frac{(2n)!}{(n+1)! (n!) }
	\label{eq:cat}
\end{eqnarray}
derivar uma implementação de $C_n$ que não calcule factoriais nenhuns.
Isto é, derivar um ciclo-\textsf{for}
\begin{hscode}\SaveRestoreHook
\column{B}{@{}>{\hspre}l<{\hspost}@{}}%
\column{E}{@{}>{\hspre}l<{\hspost}@{}}%
\>[B]{}\Varid{cat}\mathrel{=}\cdots \comp \for{\Varid{loop}}\ {\Varid{init}}\;\mathbf{where}\;\cdots {}\<[E]%
\ColumnHook
\end{hscode}\resethooks
que implemente esta função.

\begin{propriedade}
A função proposta coincidem com a definição dada:
\begin{hscode}\SaveRestoreHook
\column{B}{@{}>{\hspre}l<{\hspost}@{}}%
\column{33}{@{}>{\hspre}l<{\hspost}@{}}%
\column{E}{@{}>{\hspre}l<{\hspost}@{}}%
\>[B]{}\Varid{prop\char95 cat}\mathrel{=}(\geq \mathrm{0})\Rightarrow(\Varid{catdef}{}\<[33]%
\>[33]{}\equiv\Varid{cat}){}\<[E]%
\ColumnHook
\end{hscode}\resethooks
\end{propriedade}
%
\textbf{Sugestão}: Começar por estudar muito bem o processo de cálculo dado
no anexo \ref{sec:recmul} para o problema (semelhante) da função exponencial.


\Problema

As \bezier{curvas de Bézier}, designação dada em honra ao engenheiro
\href{https://en.wikipedia.org/wiki/Pierre_B%C3%A9zier}{Pierre Bézier},
são curvas ubíquas na área de computação gráfica, animação e modelação.
Uma curva de Bézier é uma curva paramétrica, definida por um conjunto
$\{P_0,...,P_N\}$ de pontos de controlo, onde $N$ é a ordem da curva.

\begin{figure}[h!]
  \centering
  \includegraphics[width=0.8\textwidth]{cp2021t_media/Bezier_curves.png}
  \caption{Exemplos de curvas de Bézier retirados da \bezier{ Wikipedia}.}
\end{figure}

O algoritmo de \emph{De Casteljau} é um método recursivo capaz de calcular
curvas de Bézier num ponto. Apesar de ser mais lento do que outras abordagens,
este algoritmo é numericamente mais estável, trocando velocidade por correção.

De forma sucinta, o valor de uma curva de Bézier de um só ponto $\{P_0\}$
(ordem $0$) é o próprio ponto $P_0$. O valor de uma curva de Bézier de ordem
$N$ é calculado através da interpolação linear da curva de Bézier dos primeiros
$N-1$ pontos e da curva de Bézier dos últimos $N-1$ pontos.

A interpolação linear entre 2 números, no intervalo $[0, 1]$, é dada pela
seguinte função:

\begin{hscode}\SaveRestoreHook
\column{B}{@{}>{\hspre}l<{\hspost}@{}}%
\column{3}{@{}>{\hspre}l<{\hspost}@{}}%
\column{E}{@{}>{\hspre}l<{\hspost}@{}}%
\>[B]{}\Varid{linear1d}\mathbin{::}\Q \to \Q \to \Conid{OverTime}\;\Q {}\<[E]%
\\
\>[B]{}\Varid{linear1d}\;\Varid{a}\;\Varid{b}\mathrel{=}\Varid{formula}\;\Varid{a}\;\Varid{b}\;\mathbf{where}{}\<[E]%
\\
\>[B]{}\hsindent{3}{}\<[3]%
\>[3]{}\Varid{formula}\mathbin{::}\Q \to \Q \to \Conid{Float}\to \Q {}\<[E]%
\\
\>[B]{}\hsindent{3}{}\<[3]%
\>[3]{}\Varid{formula}\;\Varid{x}\;\Varid{y}\;\Varid{t}\mathrel{=}((\mathrm{1.0}\mathbin{::}\Q )\mathbin{-}( to_\Q \;\Varid{t}))\mathbin{*}\Varid{x}\mathbin{+}( to_\Q \;\Varid{t})\mathbin{*}\Varid{y}{}\<[E]%
\ColumnHook
\end{hscode}\resethooks
%
A interpolação linear entre 2 pontos de dimensão $N$ é calculada através
da interpolação linear de cada dimensão.

O tipo de dados \ensuremath{\Conid{NPoint}} representa um ponto com $N$ dimensões.
\begin{hscode}\SaveRestoreHook
\column{B}{@{}>{\hspre}l<{\hspost}@{}}%
\column{E}{@{}>{\hspre}l<{\hspost}@{}}%
\>[B]{}\mathbf{type}\;\Conid{NPoint}\mathrel{=}[\mskip1.5mu \Q \mskip1.5mu]{}\<[E]%
\ColumnHook
\end{hscode}\resethooks
Por exemplo, um ponto de 2 dimensões e um ponto de 3 dimensões podem ser
representados, respetivamente, por:
\begin{hscode}\SaveRestoreHook
\column{B}{@{}>{\hspre}l<{\hspost}@{}}%
\column{E}{@{}>{\hspre}l<{\hspost}@{}}%
\>[B]{}\Varid{p2d}\mathrel{=}[\mskip1.5mu \mathrm{1.2},\mathrm{3.4}\mskip1.5mu]{}\<[E]%
\\
\>[B]{}\Varid{p3d}\mathrel{=}[\mskip1.5mu \mathrm{0.2},\mathrm{10.3},\mathrm{2.4}\mskip1.5mu]{}\<[E]%
\ColumnHook
\end{hscode}\resethooks
%
O tipo de dados \ensuremath{\Conid{OverTime}\;\Varid{a}} representa um termo do tipo \ensuremath{\Varid{a}} num dado instante
(dado por um \ensuremath{\Conid{Float}}).
\begin{hscode}\SaveRestoreHook
\column{B}{@{}>{\hspre}l<{\hspost}@{}}%
\column{E}{@{}>{\hspre}l<{\hspost}@{}}%
\>[B]{}\mathbf{type}\;\Conid{OverTime}\;\Varid{a}\mathrel{=}\Conid{Float}\to \Varid{a}{}\<[E]%
\ColumnHook
\end{hscode}\resethooks
%
O anexo \ref{sec:codigo} tem definida a função
    \begin{hscode}\SaveRestoreHook
\column{B}{@{}>{\hspre}l<{\hspost}@{}}%
\column{5}{@{}>{\hspre}l<{\hspost}@{}}%
\column{E}{@{}>{\hspre}l<{\hspost}@{}}%
\>[5]{}\Varid{calcLine}\mathbin{::}\Conid{NPoint}\to (\Conid{NPoint}\to \Conid{OverTime}\;\Conid{NPoint}){}\<[E]%
\ColumnHook
\end{hscode}\resethooks
que calcula a interpolação linear entre 2 pontos, e a função
    \begin{hscode}\SaveRestoreHook
\column{B}{@{}>{\hspre}l<{\hspost}@{}}%
\column{5}{@{}>{\hspre}l<{\hspost}@{}}%
\column{E}{@{}>{\hspre}l<{\hspost}@{}}%
\>[5]{}\Varid{deCasteljau}\mathbin{::}[\mskip1.5mu \Conid{NPoint}\mskip1.5mu]\to \Conid{OverTime}\;\Conid{NPoint}{}\<[E]%
\ColumnHook
\end{hscode}\resethooks
que implementa o algoritmo respectivo.

\begin{enumerate}

\item Implemente \ensuremath{\Varid{calcLine}} como um catamorfismo de listas,
testando a sua definição com a propriedade:
    \begin{propriedade} Definição alternativa.
\begin{hscode}\SaveRestoreHook
\column{B}{@{}>{\hspre}l<{\hspost}@{}}%
\column{46}{@{}>{\hspre}l<{\hspost}@{}}%
\column{E}{@{}>{\hspre}l<{\hspost}@{}}%
\>[B]{}\Varid{prop\char95 calcLine\char95 def}\mathbin{::}\Conid{NPoint}\to \Conid{NPoint}\to \Conid{Float}\to \Conid{Bool}{}\<[E]%
\\
\>[B]{}\Varid{prop\char95 calcLine\char95 def}\;\Varid{p}\;\Varid{q}\;\Varid{d}\mathrel{=}\Varid{calcLine}\;\Varid{p}\;\Varid{q}\;\Varid{d}\equiv {}\<[46]%
\>[46]{}\Varid{zipWithM}\;\Varid{linear1d}\;\Varid{p}\;\Varid{q}\;\Varid{d}{}\<[E]%
\ColumnHook
\end{hscode}\resethooks
    \end{propriedade}

\item Implemente a função \ensuremath{\Varid{deCasteljau}} como um hilomorfismo, testando agora a propriedade:
    \begin{propriedade}
      Curvas de Bézier são simétricas.
\begin{hscode}\SaveRestoreHook
\column{B}{@{}>{\hspre}l<{\hspost}@{}}%
\column{3}{@{}>{\hspre}l<{\hspost}@{}}%
\column{13}{@{}>{\hspre}l<{\hspost}@{}}%
\column{71}{@{}>{\hspre}l<{\hspost}@{}}%
\column{E}{@{}>{\hspre}l<{\hspost}@{}}%
\>[B]{}\Varid{prop\char95 bezier\char95 sym}\mathbin{::}[\mskip1.5mu [\mskip1.5mu \Q \mskip1.5mu]\mskip1.5mu]\to \Conid{Gen}\;\Conid{Bool}{}\<[E]%
\\
\>[B]{}\Varid{prop\char95 bezier\char95 sym}\;\Varid{l}\mathrel{=}\Varid{all}\;(\mathbin{<}\Delta )\comp \Varid{calc\char95 difs}\comp \Varid{bezs}\mathbin{\mathopen{\langle}\$\mathclose{\rangle}}\Varid{elements}\;\Varid{ps}\;{}\<[71]%
\>[71]{}\mathbf{where}{}\<[E]%
\\
\>[B]{}\hsindent{3}{}\<[3]%
\>[3]{}\Varid{calc\char95 difs}\mathrel{=}(\lambda (\Varid{x},\Varid{y})\to \Varid{zipWith}\;(\lambda \Varid{w}\;\Varid{v}\to \mathbf{if}\;\Varid{w}\geq \Varid{v}\;\mathbf{then}\;\Varid{w}\mathbin{-}\Varid{v}\;\mathbf{else}\;\Varid{v}\mathbin{-}\Varid{w})\;\Varid{x}\;\Varid{y}){}\<[E]%
\\
\>[B]{}\hsindent{3}{}\<[3]%
\>[3]{}\Varid{bezs}\;\Varid{t}{}\<[13]%
\>[13]{}\mathrel{=}(\Varid{deCasteljau}\;\Varid{l}\;\Varid{t},\Varid{deCasteljau}\;(\Varid{reverse}\;\Varid{l})\;( from_\Q \;(\mathrm{1}\mathbin{-}( to_\Q \;\Varid{t})))){}\<[E]%
\\
\>[B]{}\hsindent{3}{}\<[3]%
\>[3]{}\Delta \mathrel{=}\mathrm{1e{-}2}{}\<[E]%
\ColumnHook
\end{hscode}\resethooks
    \end{propriedade}

  \item Corra a função \ensuremath{\Varid{runBezier}} e aprecie o seu trabalho\footnote{%
        A representação em Gloss é uma adaptação de um
        \href{https://github.com/hrldcpr/Bezier.hs}{projeto}
        de Harold Cooper.} clicando na janela que é aberta (que contém, a verde, um ponto
        inicila) com o botão esquerdo do rato para adicionar mais pontos.
        A tecla \ensuremath{\Conid{Delete}} apaga o ponto mais recente.

\end{enumerate}

\Problema

Seja dada a fórmula que calcula a média de uma lista não vazia $x$,
\begin{equation}
avg\ x = \frac 1 k\sum_{i=1}^{k} x_i
\end{equation}
onde $k=length\ x$. Isto é, para sabermos a média de uma lista precisamos de dois catamorfismos: o que faz o somatório e o que calcula o comprimento a lista.
Contudo, é facil de ver que
\begin{quote}
	$avg\ [a]=a$
\\
	$avg (a:x) = \frac 1 {k+1}(a+\sum_{i=1}^{k} x_i) = \frac{a+k(avg\ x)}{k+1}$ para $k=length\ x$
\end{quote}
Logo $avg$ está em recursividade mútua com $length$ e o par de funções pode ser expresso por um único catamorfismo, significando que a lista apenas é percorrida uma vez.

\begin{enumerate}

\item	Recorra à lei de recursividade mútua para derivar a função
\ensuremath{\Varid{avg\char95 aux}\mathrel{=}\cata{\alt{\Varid{b}}{\Varid{q}}}} tal que
\ensuremath{\Varid{avg\char95 aux}\mathrel{=}\conj{\Varid{avg}}{\length }} em listas não vazias.

\item	Generalize o raciocínio anterior para o cálculo da média de todos os elementos de uma \LTree\ recorrendo a uma única travessia da árvore (i.e.\ catamorfismo).

\end{enumerate}
Verifique as suas funções testando a propriedade seguinte:
\begin{propriedade}
A média de uma lista não vazia e de uma \LTree\ com os mesmos elementos coincide,
a menos de um erro de 0.1 milésimas:
\begin{hscode}\SaveRestoreHook
\column{B}{@{}>{\hspre}l<{\hspost}@{}}%
\column{4}{@{}>{\hspre}l<{\hspost}@{}}%
\column{E}{@{}>{\hspre}l<{\hspost}@{}}%
\>[B]{}\Varid{prop\char95 avg}\mathbin{::}[\mskip1.5mu \Conid{Double}\mskip1.5mu]\to \Conid{Property}{}\<[E]%
\\
\>[B]{}\Varid{prop\char95 avg}\mathrel{=}\Varid{nonempty}\Rightarrow\Varid{diff}\leq\underline{\mathrm{0.000001}}\;\mathbf{where}{}\<[E]%
\\
\>[B]{}\hsindent{4}{}\<[4]%
\>[4]{}\Varid{diff}\;\Varid{l}\mathrel{=}\Varid{avg}\;\Varid{l}\mathbin{-}(\Varid{avgLTree}\comp \Varid{genLTree})\;\Varid{l}{}\<[E]%
\\
\>[B]{}\hsindent{4}{}\<[4]%
\>[4]{}\Varid{genLTree}\mathrel{=}\mathopen{[\!(}\Varid{lsplit}\mathclose{)\!]}{}\<[E]%
\\
\>[B]{}\hsindent{4}{}\<[4]%
\>[4]{}\Varid{nonempty}\mathrel{=}(\mathbin{>}[\mskip1.5mu \mskip1.5mu]){}\<[E]%
\ColumnHook
\end{hscode}\resethooks
\end{propriedade}

\Problema	(\textbf{NB}: Esta questão é \textbf{opcional} e funciona como \textbf{valorização} apenas para os alunos que desejarem fazê-la.)

\vskip 1em \noindent
Existem muitas linguagens funcionais para além do \Haskell, que é a linguagem usada neste trabalho prático. Uma delas é o \Fsharp\ da Microsoft. Na directoria \text{\ttfamily fsharp} encontram-se os módulos \Cp, \Nat\ e \LTree\ codificados em \Fsharp. O que se pede é a biblioteca \BTree\ escrita na mesma linguagem.

Modo de execução: o código que tiverem produzido nesta pergunta deve ser colocado entre o \text{\ttfamily \char92{}begin\char123{}verbatim\char125{}} e o \text{\ttfamily \char92{}end\char123{}verbatim\char125{}} da correspondente parte do anexo \ref{sec:resolucao}. Para além disso, os grupos podem demonstrar o código na oral.

\newpage

\part*{Anexos}

\appendix

\section{Como exprimir cálculos e diagramas em LaTeX/lhs2tex}
Como primeiro exemplo, estudar o texto fonte deste trabalho para obter o
efeito:\footnote{Exemplos tirados de \cite{Ol18}.}
\begin{eqnarray*}
\start
	\ensuremath{\Varid{id}\mathrel{=}\conj{\Varid{f}}{\Varid{g}}}
%
\just\equiv{ universal property }
%
        \ensuremath{\begin{lcbr}\p1\comp \Varid{id}\mathrel{=}\Varid{f}\\\p2\comp \Varid{id}\mathrel{=}\Varid{g}\end{lcbr}}
%
\just\equiv{ identity }
%
        \ensuremath{\begin{lcbr}\p1\mathrel{=}\Varid{f}\\\p2\mathrel{=}\Varid{g}\end{lcbr}}
\qed
\end{eqnarray*}

Os diagramas podem ser produzidos recorrendo à \emph{package} \LaTeX\
\href{https://ctan.org/pkg/xymatrix}{xymatrix}, por exemplo:
\begin{eqnarray*}
\xymatrix@C=2cm{
    \ensuremath{\N_0}
           \ar[d]_-{\ensuremath{\cata{\Varid{g}}}}
&
    \ensuremath{\mathrm{1}\mathbin{+}\N_0}
           \ar[d]^{\ensuremath{\Varid{id}\mathbin{+}\cata{\Varid{g}}}}
           \ar[l]_-{\ensuremath{\mathsf{in}}}
\\
     \ensuremath{\Conid{B}}
&
     \ensuremath{\mathrm{1}\mathbin{+}\Conid{B}}
           \ar[l]^-{\ensuremath{\Varid{g}}}
}
\end{eqnarray*}

\section{Programação dinâmica por recursividade múltipla}\label{sec:recmul}
Neste anexo dão-se os detalhes da resolução do Exercício \ref{ex:exp} dos apontamentos da
disciplina\footnote{Cf.\ \cite{Ol18}, página \pageref{ex:exp}.},
onde se pretende implementar um ciclo que implemente
o cálculo da aproximação até \ensuremath{\Varid{i}\mathrel{=}\Varid{n}} da função exponencial $exp\ x = e^x$,
via série de Taylor:
\begin{eqnarray}
	exp\ x
& = &
	\sum_{i=0}^{\infty} \frac {x^i} {i!}
\end{eqnarray}
Seja $e\ x\ n = \sum_{i=0}^{n} \frac {x^i} {i!}$ a função que dá essa aproximação.
É fácil de ver que \ensuremath{\Varid{e}\;\Varid{x}\;\mathrm{0}\mathrel{=}\mathrm{1}} e que $\ensuremath{\Varid{e}\;\Varid{x}\;(\Varid{n}\mathbin{+}\mathrm{1})} = \ensuremath{\Varid{e}\;\Varid{x}\;\Varid{n}} + \frac {x^{n+1}} {(n+1)!}$.
Se definirmos $\ensuremath{\Varid{h}\;\Varid{x}\;\Varid{n}} = \frac {x^{n+1}} {(n+1)!}$ teremos \ensuremath{\Varid{e}\;\Varid{x}} e \ensuremath{\Varid{h}\;\Varid{x}} em recursividade
mútua. Se repetirmos o processo para \ensuremath{\Varid{h}\;\Varid{x}\;\Varid{n}} etc obteremos no total três funções nessa mesma
situação:
\begin{hscode}\SaveRestoreHook
\column{B}{@{}>{\hspre}l<{\hspost}@{}}%
\column{E}{@{}>{\hspre}l<{\hspost}@{}}%
\>[B]{}\Varid{e}\;\Varid{x}\;\mathrm{0}\mathrel{=}\mathrm{1}{}\<[E]%
\\
\>[B]{}\Varid{e}\;\Varid{x}\;(\Varid{n}\mathbin{+}\mathrm{1})\mathrel{=}\Varid{h}\;\Varid{x}\;\Varid{n}\mathbin{+}\Varid{e}\;\Varid{x}\;\Varid{n}{}\<[E]%
\\[\blanklineskip]%
\>[B]{}\Varid{h}\;\Varid{x}\;\mathrm{0}\mathrel{=}\Varid{x}{}\<[E]%
\\
\>[B]{}\Varid{h}\;\Varid{x}\;(\Varid{n}\mathbin{+}\mathrm{1})\mathrel{=}\Varid{x}\mathbin{/}(\Varid{s}\;\Varid{n})\mathbin{*}\Varid{h}\;\Varid{x}\;\Varid{n}{}\<[E]%
\\[\blanklineskip]%
\>[B]{}\Varid{s}\;\mathrm{0}\mathrel{=}\mathrm{2}{}\<[E]%
\\
\>[B]{}\Varid{s}\;(\Varid{n}\mathbin{+}\mathrm{1})\mathrel{=}\mathrm{1}\mathbin{+}\Varid{s}\;\Varid{n}{}\<[E]%
\ColumnHook
\end{hscode}\resethooks
Segundo a \emph{regra de algibeira} descrita na página \ref{pg:regra} deste enunciado,
ter-se-á, de imediato:
\begin{hscode}\SaveRestoreHook
\column{B}{@{}>{\hspre}l<{\hspost}@{}}%
\column{6}{@{}>{\hspre}l<{\hspost}@{}}%
\column{E}{@{}>{\hspre}l<{\hspost}@{}}%
\>[B]{}\Varid{e'}\;\Varid{x}\mathrel{=}\Varid{prj}\comp \for{\Varid{loop}}\ {\Varid{init}}\;\mathbf{where}{}\<[E]%
\\
\>[B]{}\hsindent{6}{}\<[6]%
\>[6]{}\Varid{init}\mathrel{=}(\mathrm{1},\Varid{x},\mathrm{2}){}\<[E]%
\\
\>[B]{}\hsindent{6}{}\<[6]%
\>[6]{}\Varid{loop}\;(\Varid{e},\Varid{h},\Varid{s})\mathrel{=}(\Varid{h}\mathbin{+}\Varid{e},\Varid{x}\mathbin{/}\Varid{s}\mathbin{*}\Varid{h},\mathrm{1}\mathbin{+}\Varid{s}){}\<[E]%
\\
\>[B]{}\hsindent{6}{}\<[6]%
\>[6]{}\Varid{prj}\;(\Varid{e},\Varid{h},\Varid{s})\mathrel{=}\Varid{e}{}\<[E]%
\ColumnHook
\end{hscode}\resethooks

\section{Código fornecido}\label{sec:codigo}

\subsection*{Problema 1}

\begin{hscode}\SaveRestoreHook
\column{B}{@{}>{\hspre}l<{\hspost}@{}}%
\column{E}{@{}>{\hspre}l<{\hspost}@{}}%
\>[B]{}\Varid{expd}\mathbin{::}\Conid{Floating}\;\Varid{a}\Rightarrow \Varid{a}\to \Varid{a}{}\<[E]%
\\
\>[B]{}\Varid{expd}\mathrel{=}\Varid{\Conid{Prelude}.exp}{}\<[E]%
\\[\blanklineskip]%
\>[B]{}\mathbf{type}\;\Conid{OutExpAr}\;\Varid{a}\mathrel{=}()+(\Varid{a}+((\Conid{BinOp},(\Conid{ExpAr}\;\Varid{a},\Conid{ExpAr}\;\Varid{a}))+(\Conid{UnOp},\Conid{ExpAr}\;\Varid{a}))){}\<[E]%
\ColumnHook
\end{hscode}\resethooks

\subsection*{Problema 2}
Definição da série de Catalan usando factoriais (\ref{eq:cat}):
\begin{hscode}\SaveRestoreHook
\column{B}{@{}>{\hspre}l<{\hspost}@{}}%
\column{E}{@{}>{\hspre}l<{\hspost}@{}}%
\>[B]{}\Varid{catdef}\;\Varid{n}\mathrel{=}{(\mathrm{2}\mathbin{*}\Varid{n})!}\div ({(\Varid{n}\mathbin{+}\mathrm{1})!}\mathbin{*}{\Varid{n}!}){}\<[E]%
\ColumnHook
\end{hscode}\resethooks
Oráculo para inspecção dos primeiros 26 números de Catalan\footnote{Fonte:
\catalan{Wikipedia}.}:
\begin{hscode}\SaveRestoreHook
\column{B}{@{}>{\hspre}l<{\hspost}@{}}%
\column{5}{@{}>{\hspre}l<{\hspost}@{}}%
\column{E}{@{}>{\hspre}l<{\hspost}@{}}%
\>[B]{}\Varid{oracle}\mathrel{=}[\mskip1.5mu {}\<[E]%
\\
\>[B]{}\hsindent{5}{}\<[5]%
\>[5]{}\mathrm{1},\mathrm{1},\mathrm{2},\mathrm{5},\mathrm{14},\mathrm{42},\mathrm{132},\mathrm{429},\mathrm{1430},\mathrm{4862},\mathrm{16796},\mathrm{58786},\mathrm{208012},\mathrm{742900},\mathrm{2674440},\mathrm{9694845},{}\<[E]%
\\
\>[B]{}\hsindent{5}{}\<[5]%
\>[5]{}\mathrm{35357670},\mathrm{129644790},\mathrm{477638700},\mathrm{1767263190},\mathrm{6564120420},\mathrm{24466267020},{}\<[E]%
\\
\>[B]{}\hsindent{5}{}\<[5]%
\>[5]{}\mathrm{91482563640},\mathrm{343059613650},\mathrm{1289904147324},\mathrm{4861946401452}{}\<[E]%
\\
\>[B]{}\hsindent{5}{}\<[5]%
\>[5]{}\mskip1.5mu]{}\<[E]%
\ColumnHook
\end{hscode}\resethooks

\subsection*{Problema 3}
Algoritmo:
\begin{hscode}\SaveRestoreHook
\column{B}{@{}>{\hspre}l<{\hspost}@{}}%
\column{3}{@{}>{\hspre}l<{\hspost}@{}}%
\column{E}{@{}>{\hspre}l<{\hspost}@{}}%
\>[B]{}\Varid{deCasteljau}\mathbin{::}[\mskip1.5mu \Conid{NPoint}\mskip1.5mu]\to \Conid{OverTime}\;\Conid{NPoint}{}\<[E]%
\\
\>[B]{}\Varid{deCasteljau}\;[\mskip1.5mu \mskip1.5mu]\mathrel{=}\Varid{nil}{}\<[E]%
\\
\>[B]{}\Varid{deCasteljau}\;[\mskip1.5mu \Varid{p}\mskip1.5mu]\mathrel{=}\underline{\Varid{p}}{}\<[E]%
\\
\>[B]{}\Varid{deCasteljau}\;\Varid{l}\mathrel{=}\lambda \Varid{pt}\to (\Varid{calcLine}\;(\Varid{p}\;\Varid{pt})\;(\Varid{q}\;\Varid{pt}))\;\Varid{pt}\;\mathbf{where}{}\<[E]%
\\
\>[B]{}\hsindent{3}{}\<[3]%
\>[3]{}\Varid{p}\mathrel{=}\Varid{deCasteljau}\;(\Varid{init}\;\Varid{l}){}\<[E]%
\\
\>[B]{}\hsindent{3}{}\<[3]%
\>[3]{}\Varid{q}\mathrel{=}\Varid{deCasteljau}\;(\Varid{tail}\;\Varid{l}){}\<[E]%
\ColumnHook
\end{hscode}\resethooks
Função auxiliar:
\begin{hscode}\SaveRestoreHook
\column{B}{@{}>{\hspre}l<{\hspost}@{}}%
\column{4}{@{}>{\hspre}l<{\hspost}@{}}%
\column{8}{@{}>{\hspre}l<{\hspost}@{}}%
\column{15}{@{}>{\hspre}l<{\hspost}@{}}%
\column{E}{@{}>{\hspre}l<{\hspost}@{}}%
\>[B]{}\Varid{calcLine}\mathbin{::}\Conid{NPoint}\to (\Conid{NPoint}\to \Conid{OverTime}\;\Conid{NPoint}){}\<[E]%
\\
\>[B]{}\Varid{calcLine}\;[\mskip1.5mu \mskip1.5mu]\mathrel{=}\underline{\Varid{nil}}{}\<[E]%
\\
\>[B]{}\Varid{calcLine}\;(\Varid{p}\mathbin{:}\Varid{x})\mathrel{=}\overline{\Varid{g}}\;\Varid{p}\;(\Varid{calcLine}\;\Varid{x})\;\mathbf{where}{}\<[E]%
\\
\>[B]{}\hsindent{4}{}\<[4]%
\>[4]{}\Varid{g}\mathbin{::}(\Q ,\Conid{NPoint}\to \Conid{OverTime}\;\Conid{NPoint})\to (\Conid{NPoint}\to \Conid{OverTime}\;\Conid{NPoint}){}\<[E]%
\\
\>[B]{}\hsindent{4}{}\<[4]%
\>[4]{}\Varid{g}\;(\Varid{d},\Varid{f})\;\Varid{l}\mathrel{=}\mathbf{case}\;\Varid{l}\;\mathbf{of}{}\<[E]%
\\
\>[4]{}\hsindent{4}{}\<[8]%
\>[8]{}[\mskip1.5mu \mskip1.5mu]{}\<[15]%
\>[15]{}\to \Varid{nil}{}\<[E]%
\\
\>[4]{}\hsindent{4}{}\<[8]%
\>[8]{}(\Varid{x}\mathbin{:}\Varid{xs})\to \lambda \Varid{z}\to \Varid{concat}\mathbin{\$}(\Varid{sequenceA}\;[\mskip1.5mu \Varid{singl}\comp \Varid{linear1d}\;\Varid{d}\;\Varid{x},\Varid{f}\;\Varid{xs}\mskip1.5mu])\;\Varid{z}{}\<[E]%
\ColumnHook
\end{hscode}\resethooks
2D:
\begin{hscode}\SaveRestoreHook
\column{B}{@{}>{\hspre}l<{\hspost}@{}}%
\column{E}{@{}>{\hspre}l<{\hspost}@{}}%
\>[B]{}\Varid{bezier2d}\mathbin{::}[\mskip1.5mu \Conid{NPoint}\mskip1.5mu]\to \Conid{OverTime}\;(\Conid{Float},\Conid{Float}){}\<[E]%
\\
\>[B]{}\Varid{bezier2d}\;[\mskip1.5mu \mskip1.5mu]\mathrel{=}\underline{(\mathrm{0},\mathrm{0})}{}\<[E]%
\\
\>[B]{}\Varid{bezier2d}\;\Varid{l}\mathrel{=}\lambda \Varid{z}\to ( from_\Q \times from_\Q )\comp (\lambda [\mskip1.5mu \Varid{x},\Varid{y}\mskip1.5mu]\to (\Varid{x},\Varid{y}))\mathbin{\$}((\Varid{deCasteljau}\;\Varid{l})\;\Varid{z}){}\<[E]%
\ColumnHook
\end{hscode}\resethooks
Modelo:
\begin{hscode}\SaveRestoreHook
\column{B}{@{}>{\hspre}l<{\hspost}@{}}%
\column{3}{@{}>{\hspre}l<{\hspost}@{}}%
\column{5}{@{}>{\hspre}l<{\hspost}@{}}%
\column{20}{@{}>{\hspre}l<{\hspost}@{}}%
\column{E}{@{}>{\hspre}l<{\hspost}@{}}%
\>[B]{}\mathbf{data}\;\Conid{World}\mathrel{=}\Conid{World}\;\{\mskip1.5mu \Varid{points}\mathbin{::}[\mskip1.5mu \Conid{NPoint}\mskip1.5mu]{}\<[E]%
\\
\>[B]{}\hsindent{20}{}\<[20]%
\>[20]{},\Varid{time}\mathbin{::}\Conid{Float}{}\<[E]%
\\
\>[B]{}\hsindent{20}{}\<[20]%
\>[20]{}\mskip1.5mu\}{}\<[E]%
\\
\>[B]{}\Varid{initW}\mathbin{::}\Conid{World}{}\<[E]%
\\
\>[B]{}\Varid{initW}\mathrel{=}\Conid{World}\;[\mskip1.5mu \mskip1.5mu]\;\mathrm{0}{}\<[E]%
\\[\blanklineskip]%
\>[B]{}\Varid{tick}\mathbin{::}\Conid{Float}\to \Conid{World}\to \Conid{World}{}\<[E]%
\\
\>[B]{}\Varid{tick}\;\Varid{dt}\;\Varid{world}\mathrel{=}\Varid{world}\;\{\mskip1.5mu \Varid{time}\mathrel{=}(\Varid{time}\;\Varid{world})\mathbin{+}\Varid{dt}\mskip1.5mu\}{}\<[E]%
\\[\blanklineskip]%
\>[B]{}\Varid{actions}\mathbin{::}\Conid{Event}\to \Conid{World}\to \Conid{World}{}\<[E]%
\\
\>[B]{}\Varid{actions}\;(\Conid{EventKey}\;(\Conid{MouseButton}\;\Conid{LeftButton})\;\Conid{Down}\;\anonymous \;\Varid{p})\;\Varid{world}\mathrel{=}{}\<[E]%
\\
\>[B]{}\hsindent{3}{}\<[3]%
\>[3]{}\Varid{world}\;\{\mskip1.5mu \Varid{points}\mathrel{=}(\Varid{points}\;\Varid{world})\plus [\mskip1.5mu (\lambda (\Varid{x},\Varid{y})\to \map \; to_\Q \;[\mskip1.5mu \Varid{x},\Varid{y}\mskip1.5mu])\;\Varid{p}\mskip1.5mu]\mskip1.5mu\}{}\<[E]%
\\
\>[B]{}\Varid{actions}\;(\Conid{EventKey}\;(\Conid{SpecialKey}\;\Conid{KeyDelete})\;\Conid{Down}\;\anonymous \;\anonymous )\;\Varid{world}\mathrel{=}{}\<[E]%
\\
\>[B]{}\hsindent{5}{}\<[5]%
\>[5]{}\Varid{world}\;\{\mskip1.5mu \Varid{points}\mathrel{=}\Varid{cond}\;(\equiv [\mskip1.5mu \mskip1.5mu])\;\Varid{id}\;\Varid{init}\;(\Varid{points}\;\Varid{world})\mskip1.5mu\}{}\<[E]%
\\
\>[B]{}\Varid{actions}\;\anonymous \;\Varid{world}\mathrel{=}\Varid{world}{}\<[E]%
\\[\blanklineskip]%
\>[B]{}\Varid{scaleTime}\mathbin{::}\Conid{World}\to \Conid{Float}{}\<[E]%
\\
\>[B]{}\Varid{scaleTime}\;\Varid{w}\mathrel{=}(\mathrm{1}\mathbin{+}\Varid{cos}\;(\Varid{time}\;\Varid{w}))\mathbin{/}\mathrm{2}{}\<[E]%
\\[\blanklineskip]%
\>[B]{}\Varid{bezier2dAtTime}\mathbin{::}\Conid{World}\to (\Conid{Float},\Conid{Float}){}\<[E]%
\\
\>[B]{}\Varid{bezier2dAtTime}\;\Varid{w}\mathrel{=}(\Varid{bezier2dAt}\;\Varid{w})\;(\Varid{scaleTime}\;\Varid{w}){}\<[E]%
\\[\blanklineskip]%
\>[B]{}\Varid{bezier2dAt}\mathbin{::}\Conid{World}\to \Conid{OverTime}\;(\Conid{Float},\Conid{Float}){}\<[E]%
\\
\>[B]{}\Varid{bezier2dAt}\;\Varid{w}\mathrel{=}\Varid{bezier2d}\;(\Varid{points}\;\Varid{w}){}\<[E]%
\\[\blanklineskip]%
\>[B]{}\Varid{thicCirc}\mathbin{::}\Conid{Picture}{}\<[E]%
\\
\>[B]{}\Varid{thicCirc}\mathrel{=}\Conid{ThickCircle}\;\mathrm{4}\;\mathrm{10}{}\<[E]%
\\[\blanklineskip]%
\>[B]{}\Varid{ps}\mathbin{::}[\mskip1.5mu \Conid{Float}\mskip1.5mu]{}\<[E]%
\\
\>[B]{}\Varid{ps}\mathrel{=}\map \; from_\Q \;\Varid{ps'}\;\mathbf{where}{}\<[E]%
\\
\>[B]{}\hsindent{3}{}\<[3]%
\>[3]{}\Varid{ps'}\mathbin{::}[\mskip1.5mu \Q \mskip1.5mu]{}\<[E]%
\\
\>[B]{}\hsindent{3}{}\<[3]%
\>[3]{}\Varid{ps'}\mathrel{=}[\mskip1.5mu \mathrm{0},\mathrm{0.01}\mathinner{\ldotp\ldotp}\mathrm{1}\mskip1.5mu]\mbox{\onelinecomment  interval}{}\<[E]%
\ColumnHook
\end{hscode}\resethooks
Gloss:
\begin{hscode}\SaveRestoreHook
\column{B}{@{}>{\hspre}l<{\hspost}@{}}%
\column{3}{@{}>{\hspre}l<{\hspost}@{}}%
\column{E}{@{}>{\hspre}l<{\hspost}@{}}%
\>[B]{}\Varid{picture}\mathbin{::}\Conid{World}\to \Conid{Picture}{}\<[E]%
\\
\>[B]{}\Varid{picture}\;\Varid{world}\mathrel{=}\Conid{Pictures}{}\<[E]%
\\
\>[B]{}\hsindent{3}{}\<[3]%
\>[3]{}[\mskip1.5mu \Varid{animateBezier}\;(\Varid{scaleTime}\;\Varid{world})\;(\Varid{points}\;\Varid{world}){}\<[E]%
\\
\>[B]{}\hsindent{3}{}\<[3]%
\>[3]{},\Conid{Color}\;\Varid{white}\comp \Conid{Line}\comp \map \;(\Varid{bezier2dAt}\;\Varid{world})\mathbin{\$}\Varid{ps}{}\<[E]%
\\
\>[B]{}\hsindent{3}{}\<[3]%
\>[3]{},\Conid{Color}\;\Varid{blue}\comp \Conid{Pictures}\mathbin{\$}[\mskip1.5mu \Conid{Translate}\;( from_\Q \;\Varid{x})\;( from_\Q \;\Varid{y})\;\Varid{thicCirc}\mid [\mskip1.5mu \Varid{x},\Varid{y}\mskip1.5mu]\leftarrow \Varid{points}\;\Varid{world}\mskip1.5mu]{}\<[E]%
\\
\>[B]{}\hsindent{3}{}\<[3]%
\>[3]{},\Conid{Color}\;\Varid{green}\mathbin{\$}\Conid{Translate}\;\Varid{cx}\;\Varid{cy}\;\Varid{thicCirc}{}\<[E]%
\\
\>[B]{}\hsindent{3}{}\<[3]%
\>[3]{}\mskip1.5mu]\;\mathbf{where}{}\<[E]%
\\
\>[B]{}\hsindent{3}{}\<[3]%
\>[3]{}(\Varid{cx},\Varid{cy})\mathrel{=}\Varid{bezier2dAtTime}\;\Varid{world}{}\<[E]%
\ColumnHook
\end{hscode}\resethooks
Animação:
\begin{hscode}\SaveRestoreHook
\column{B}{@{}>{\hspre}l<{\hspost}@{}}%
\column{3}{@{}>{\hspre}l<{\hspost}@{}}%
\column{E}{@{}>{\hspre}l<{\hspost}@{}}%
\>[B]{}\Varid{animateBezier}\mathbin{::}\Conid{Float}\to [\mskip1.5mu \Conid{NPoint}\mskip1.5mu]\to \Conid{Picture}{}\<[E]%
\\
\>[B]{}\Varid{animateBezier}\;\anonymous \;[\mskip1.5mu \mskip1.5mu]\mathrel{=}\Conid{Blank}{}\<[E]%
\\
\>[B]{}\Varid{animateBezier}\;\anonymous \;[\mskip1.5mu \anonymous \mskip1.5mu]\mathrel{=}\Conid{Blank}{}\<[E]%
\\
\>[B]{}\Varid{animateBezier}\;\Varid{t}\;\Varid{l}\mathrel{=}\Conid{Pictures}{}\<[E]%
\\
\>[B]{}\hsindent{3}{}\<[3]%
\>[3]{}[\mskip1.5mu \Varid{animateBezier}\;\Varid{t}\;(\Varid{init}\;\Varid{l}){}\<[E]%
\\
\>[B]{}\hsindent{3}{}\<[3]%
\>[3]{},\Varid{animateBezier}\;\Varid{t}\;(\Varid{tail}\;\Varid{l}){}\<[E]%
\\
\>[B]{}\hsindent{3}{}\<[3]%
\>[3]{},\Conid{Color}\;\Varid{red}\comp \Conid{Line}\mathbin{\$}[\mskip1.5mu \Varid{a},\Varid{b}\mskip1.5mu]{}\<[E]%
\\
\>[B]{}\hsindent{3}{}\<[3]%
\>[3]{},\Conid{Color}\;\Varid{orange}\mathbin{\$}\Conid{Translate}\;\Varid{ax}\;\Varid{ay}\;\Varid{thicCirc}{}\<[E]%
\\
\>[B]{}\hsindent{3}{}\<[3]%
\>[3]{},\Conid{Color}\;\Varid{orange}\mathbin{\$}\Conid{Translate}\;\Varid{bx}\;\Varid{by}\;\Varid{thicCirc}{}\<[E]%
\\
\>[B]{}\hsindent{3}{}\<[3]%
\>[3]{}\mskip1.5mu]\;\mathbf{where}{}\<[E]%
\\
\>[B]{}\hsindent{3}{}\<[3]%
\>[3]{}\Varid{a}\mathord{@}(\Varid{ax},\Varid{ay})\mathrel{=}\Varid{bezier2d}\;(\Varid{init}\;\Varid{l})\;\Varid{t}{}\<[E]%
\\
\>[B]{}\hsindent{3}{}\<[3]%
\>[3]{}\Varid{b}\mathord{@}(\Varid{bx},\Varid{by})\mathrel{=}\Varid{bezier2d}\;(\Varid{tail}\;\Varid{l})\;\Varid{t}{}\<[E]%
\ColumnHook
\end{hscode}\resethooks
Propriedades e \emph{main}:
\begin{hscode}\SaveRestoreHook
\column{B}{@{}>{\hspre}l<{\hspost}@{}}%
\column{3}{@{}>{\hspre}l<{\hspost}@{}}%
\column{53}{@{}>{\hspre}l<{\hspost}@{}}%
\column{E}{@{}>{\hspre}l<{\hspost}@{}}%
\>[B]{}\Varid{runBezier}\mathbin{::}\fun{IO}\;(){}\<[E]%
\\
\>[B]{}\Varid{runBezier}\mathrel{=}\Varid{play}\;(\Conid{InWindow}\;\text{\ttfamily \char34 Bézier\char34}\;(\mathrm{600},\mathrm{600})\;(\mathrm{0},{}\<[53]%
\>[53]{}\mathrm{0})){}\<[E]%
\\
\>[B]{}\hsindent{3}{}\<[3]%
\>[3]{}\Varid{black}\;\mathrm{50}\;\Varid{initW}\;\Varid{picture}\;\Varid{actions}\;\Varid{tick}{}\<[E]%
\\[\blanklineskip]%
\>[B]{}\Varid{runBezierSym}\mathbin{::}\fun{IO}\;(){}\<[E]%
\\
\>[B]{}\Varid{runBezierSym}\mathrel{=}\Varid{quickCheckWith}\;(\Varid{stdArgs}\;\{\mskip1.5mu \Varid{maxSize}\mathrel{=}\mathrm{20},\Varid{maxSuccess}\mathrel{=}\mathrm{200}\mskip1.5mu\})\;\Varid{prop\char95 bezier\char95 sym}{}\<[E]%
\ColumnHook
\end{hscode}\resethooks

Compilação e execução dentro do interpretador:\footnote{Pode ser útil em testes
envolvendo \gloss{Gloss}. Nesse caso, o teste em causa deve fazer parte de uma função
\ensuremath{\Varid{main}}.}
\begin{hscode}\SaveRestoreHook
\column{B}{@{}>{\hspre}l<{\hspost}@{}}%
\column{E}{@{}>{\hspre}l<{\hspost}@{}}%
\>[B]{}\Varid{main}\mathrel{=}\Varid{runBezier}{}\<[E]%
\\[\blanklineskip]%
\>[B]{}\Varid{run}\mathrel{=}\mathbf{do}\;\{\mskip1.5mu \Varid{system}\;\text{\ttfamily \char34 ghc~cp2021t\char34};\Varid{system}\;\text{\ttfamily \char34 ./cp2021t\char34}\mskip1.5mu\}{}\<[E]%
\ColumnHook
\end{hscode}\resethooks

\subsection*{QuickCheck}
Código para geração de testes:
\begin{hscode}\SaveRestoreHook
\column{B}{@{}>{\hspre}l<{\hspost}@{}}%
\column{3}{@{}>{\hspre}l<{\hspost}@{}}%
\column{5}{@{}>{\hspre}l<{\hspost}@{}}%
\column{11}{@{}>{\hspre}l<{\hspost}@{}}%
\column{E}{@{}>{\hspre}l<{\hspost}@{}}%
\>[B]{}\mathbf{instance}\;\Conid{Arbitrary}\;\Conid{UnOp}\;\mathbf{where}{}\<[E]%
\\
\>[B]{}\hsindent{3}{}\<[3]%
\>[3]{}\Varid{arbitrary}\mathrel{=}\Varid{elements}\;[\mskip1.5mu \Conid{Negate},\Conid{E}\mskip1.5mu]{}\<[E]%
\\[\blanklineskip]%
\>[B]{}\mathbf{instance}\;\Conid{Arbitrary}\;\Conid{BinOp}\;\mathbf{where}{}\<[E]%
\\
\>[B]{}\hsindent{3}{}\<[3]%
\>[3]{}\Varid{arbitrary}\mathrel{=}\Varid{elements}\;[\mskip1.5mu \Conid{Sum},\Conid{Product}\mskip1.5mu]{}\<[E]%
\\[\blanklineskip]%
\>[B]{}\mathbf{instance}\;(\Conid{Arbitrary}\;\Varid{a})\Rightarrow \Conid{Arbitrary}\;(\Conid{ExpAr}\;\Varid{a})\;\mathbf{where}{}\<[E]%
\\
\>[B]{}\hsindent{3}{}\<[3]%
\>[3]{}\Varid{arbitrary}\mathrel{=}\mathbf{do}{}\<[E]%
\\
\>[3]{}\hsindent{2}{}\<[5]%
\>[5]{}\Varid{binop}\leftarrow \Varid{arbitrary}{}\<[E]%
\\
\>[3]{}\hsindent{2}{}\<[5]%
\>[5]{}\Varid{unop}{}\<[11]%
\>[11]{}\leftarrow \Varid{arbitrary}{}\<[E]%
\\
\>[3]{}\hsindent{2}{}\<[5]%
\>[5]{}\Varid{exp1}{}\<[11]%
\>[11]{}\leftarrow \Varid{arbitrary}{}\<[E]%
\\
\>[3]{}\hsindent{2}{}\<[5]%
\>[5]{}\Varid{exp2}{}\<[11]%
\>[11]{}\leftarrow \Varid{arbitrary}{}\<[E]%
\\
\>[3]{}\hsindent{2}{}\<[5]%
\>[5]{}\Varid{a}{}\<[11]%
\>[11]{}\leftarrow \Varid{arbitrary}{}\<[E]%
\\[\blanklineskip]%
\>[3]{}\hsindent{2}{}\<[5]%
\>[5]{}\Varid{frequency}\comp \map \;(\Varid{id}\times\Varid{pure})\mathbin{\$}[\mskip1.5mu (\mathrm{20},\Conid{X}),(\mathrm{15},\Conid{N}\;\Varid{a}),(\mathrm{35},\Conid{Bin}\;\Varid{binop}\;\Varid{exp1}\;\Varid{exp2}),(\mathrm{30},\Conid{Un}\;\Varid{unop}\;\Varid{exp1})\mskip1.5mu]{}\<[E]%
\\[\blanklineskip]%
\>[B]{}\mathbf{infixr}\;\mathrm{5}\mathbin{\stackrel{\mathrm{?}}{=}}{}\<[E]%
\\
\>[B]{}(\mathbin{\stackrel{\mathrm{?}}{=}})\mathbin{::}\Conid{Real}\;\Varid{a}\Rightarrow \Varid{a}\to \Varid{a}\to \Conid{Bool}{}\<[E]%
\\
\>[B]{}(\mathbin{\stackrel{\mathrm{?}}{=}})\;\Varid{x}\;\Varid{y}\mathrel{=}( to_\Q \;\Varid{x})\equiv ( to_\Q \;\Varid{y}){}\<[E]%
\ColumnHook
\end{hscode}\resethooks

\subsection*{Outras funções auxiliares}
%----------------- Outras definições auxiliares -------------------------------------------%
Lógicas:
\begin{hscode}\SaveRestoreHook
\column{B}{@{}>{\hspre}l<{\hspost}@{}}%
\column{E}{@{}>{\hspre}l<{\hspost}@{}}%
\>[B]{}\mathbf{infixr}\;\mathrm{0}\Rightarrow{}\<[E]%
\\
\>[B]{}(\Rightarrow)\mathbin{::}(\Conid{Testable}\;\Varid{prop})\Rightarrow (\Varid{a}\to \Conid{Bool})\to (\Varid{a}\to \Varid{prop})\to \Varid{a}\to \Conid{Property}{}\<[E]%
\\
\>[B]{}\Varid{p}\Rightarrow\Varid{f}\mathrel{=}\lambda \Varid{a}\to \Varid{p}\;\Varid{a}\Rightarrow\Varid{f}\;\Varid{a}{}\<[E]%
\\[\blanklineskip]%
\>[B]{}\mathbf{infixr}\;\mathrm{0}\Leftrightarrow{}\<[E]%
\\
\>[B]{}(\Leftrightarrow)\mathbin{::}(\Varid{a}\to \Conid{Bool})\to (\Varid{a}\to \Conid{Bool})\to \Varid{a}\to \Conid{Property}{}\<[E]%
\\
\>[B]{}\Varid{p}\Leftrightarrow\Varid{f}\mathrel{=}\lambda \Varid{a}\to (\Varid{p}\;\Varid{a}\Rightarrow\Varid{property}\;(\Varid{f}\;\Varid{a}))\mathbin{.\&\&.}(\Varid{f}\;\Varid{a}\Rightarrow\Varid{property}\;(\Varid{p}\;\Varid{a})){}\<[E]%
\\[\blanklineskip]%
\>[B]{}\mathbf{infixr}\;\mathrm{4}\equiv{}\<[E]%
\\
\>[B]{}(\equiv)\mathbin{::}\Conid{Eq}\;\Varid{b}\Rightarrow (\Varid{a}\to \Varid{b})\to (\Varid{a}\to \Varid{b})\to (\Varid{a}\to \Conid{Bool}){}\<[E]%
\\
\>[B]{}\Varid{f}\equiv\Varid{g}\mathrel{=}\lambda \Varid{a}\to \Varid{f}\;\Varid{a}\equiv \Varid{g}\;\Varid{a}{}\<[E]%
\\[\blanklineskip]%
\>[B]{}\mathbf{infixr}\;\mathrm{4}\leq{}\<[E]%
\\
\>[B]{}(\leq)\mathbin{::}\Conid{Ord}\;\Varid{b}\Rightarrow (\Varid{a}\to \Varid{b})\to (\Varid{a}\to \Varid{b})\to (\Varid{a}\to \Conid{Bool}){}\<[E]%
\\
\>[B]{}\Varid{f}\leq\Varid{g}\mathrel{=}\lambda \Varid{a}\to \Varid{f}\;\Varid{a}\leq \Varid{g}\;\Varid{a}{}\<[E]%
\\[\blanklineskip]%
\>[B]{}\mathbf{infixr}\;\mathrm{4}\wedge{}\<[E]%
\\
\>[B]{}(\wedge)\mathbin{::}(\Varid{a}\to \Conid{Bool})\to (\Varid{a}\to \Conid{Bool})\to (\Varid{a}\to \Conid{Bool}){}\<[E]%
\\
\>[B]{}\Varid{f}\wedge\Varid{g}\mathrel{=}\lambda \Varid{a}\to ((\Varid{f}\;\Varid{a})\mathrel{\wedge}(\Varid{g}\;\Varid{a})){}\<[E]%
\ColumnHook
\end{hscode}\resethooks

%----------------- Soluções dos alunos -----------------------------------------%

\section{Soluções dos alunos}\label{sec:resolucao}
Os alunos devem colocar neste anexo as suas soluções para os exercícios
propostos, de acordo com o "layout" que se fornece. Não podem ser
alterados os nomes ou tipos das funções dadas, mas pode ser adicionado
texto, disgramas e/ou outras funções auxiliares que sejam necessárias.

Valoriza-se a escrita de \emph{pouco} código que corresponda a soluções
simples e elegantes.

\subsection*{Problema 1} \label{pg:P1}
São dadas:
\begin{hscode}\SaveRestoreHook
\column{B}{@{}>{\hspre}l<{\hspost}@{}}%
\column{E}{@{}>{\hspre}l<{\hspost}@{}}%
\>[B]{}\Varid{cataExpAr}\;\Varid{g}\mathrel{=}\Varid{g}\comp \Varid{recExpAr}\;(\Varid{cataExpAr}\;\Varid{g})\comp \Varid{outExpAr}{}\<[E]%
\\
\>[B]{}\Varid{anaExpAr}\;\Varid{g}\mathrel{=}\Varid{inExpAr}\comp \Varid{recExpAr}\;(\Varid{anaExpAr}\;\Varid{g})\comp \Varid{g}{}\<[E]%
\\
\>[B]{}\Varid{hyloExpAr}\;\Varid{h}\;\Varid{g}\mathrel{=}\Varid{cataExpAr}\;\Varid{h}\comp \Varid{anaExpAr}\;\Varid{g}{}\<[E]%
\\[\blanklineskip]%
\>[B]{}\Varid{eval\char95 exp}\mathbin{::}\Conid{Floating}\;\Varid{a}\Rightarrow \Varid{a}\to (\Conid{ExpAr}\;\Varid{a})\to \Varid{a}{}\<[E]%
\\
\>[B]{}\Varid{eval\char95 exp}\;\Varid{a}\mathrel{=}\Varid{cataExpAr}\;(\Varid{g\char95 eval\char95 exp}\;\Varid{a}){}\<[E]%
\\[\blanklineskip]%
\>[B]{}\Varid{optmize\char95 eval}\mathbin{::}(\Conid{Floating}\;\Varid{a},\Conid{Eq}\;\Varid{a})\Rightarrow \Varid{a}\to (\Conid{ExpAr}\;\Varid{a})\to \Varid{a}{}\<[E]%
\\
\>[B]{}\Varid{optmize\char95 eval}\;\Varid{a}\mathrel{=}\Varid{hyloExpAr}\;(\Varid{gopt}\;\Varid{a})\;\Varid{clean}{}\<[E]%
\\[\blanklineskip]%
\>[B]{}\Varid{sd}\mathbin{::}\Conid{Floating}\;\Varid{a}\Rightarrow \Conid{ExpAr}\;\Varid{a}\to \Conid{ExpAr}\;\Varid{a}{}\<[E]%
\\
\>[B]{}\Varid{sd}\mathrel{=}\p2\comp \Varid{cataExpAr}\;\Varid{sd\char95 gen}{}\<[E]%
\\[\blanklineskip]%
\>[B]{}\Varid{ad}\mathbin{::}\Conid{Floating}\;\Varid{a}\Rightarrow \Varid{a}\to \Conid{ExpAr}\;\Varid{a}\to \Varid{a}{}\<[E]%
\\
\>[B]{}\Varid{ad}\;\Varid{v}\mathrel{=}\p2\comp \Varid{cataExpAr}\;(\Varid{ad\char95 gen}\;\Varid{v}){}\<[E]%
\ColumnHook
\end{hscode}\resethooks

\subsubsection{Alínea 1} \label{pg:P1.1}

Para resolver esta alínea, podemos começar por inferir o \ensuremath{\Varid{outExpAr}} através da propriedade do isomorfismo:

\begin{eqnarray*}
\start
	\ensuremath{\Varid{outExpAr}\comp \Varid{inExpAr}\mathrel{=}\Varid{id}}
%
\just\equiv{ \ensuremath{\Varid{inExpAr}}:= \ensuremath{[\mskip1.5mu \underline{\Conid{X}},\Varid{num\char95 ops}\mskip1.5mu]} }
%
  \ensuremath{\Varid{outExpAr}\comp [\mskip1.5mu \underline{\Conid{X}},\Varid{num\char95 ops}\mskip1.5mu]\mathrel{=}\Varid{id}}
%
\just\equiv{ Fusão-+(20) \& Functor-id(26) }
%
  \ensuremath{[\mskip1.5mu \Varid{outExpAr}\comp \underline{\Conid{X}},\Varid{outExpAr}\comp \Varid{num\char95 ops}\mskip1.5mu]\mathrel{=}[\mskip1.5mu \Varid{i\char95 1},\Varid{i\char95 2}\mskip1.5mu]}
%
\just\equiv{ \ensuremath{\Varid{num\char95 ops}}:= \ensuremath{[\mskip1.5mu \Conid{N},\Varid{ops}\mskip1.5mu]} \& \ensuremath{\Varid{ops}}:= \ensuremath{[\mskip1.5mu \Varid{bin},\uncurry{\Conid{Un}}\mskip1.5mu]} \& Eq-+(27) }
%
  \begin{lcbr}
    \ensuremath{\Varid{outExpAr}\comp \underline{\Conid{X}}\mathrel{=}\Varid{i\char95 1}}\\
    \ensuremath{\Varid{outExpAr}\comp [\mskip1.5mu \Conid{N},[\mskip1.5mu \Varid{bin},\uncurry{\Conid{Un}}\mskip1.5mu]\mskip1.5mu]\mathrel{=}\Varid{i\char95 2}}
  \end{lcbr}
%
\just\equiv{ Fusão-+(20) }
%
  \begin{lcbr}
    \ensuremath{\Varid{outExpAr}\comp \underline{\Conid{X}}\mathrel{=}\Varid{i\char95 1}}\\
    \ensuremath{[\mskip1.5mu \Varid{outExpAr}\comp \Conid{N},\Varid{outExpAr}\comp [\mskip1.5mu \Varid{bin},\uncurry{\Conid{Un}}\mskip1.5mu]\mskip1.5mu]\mathrel{=}\Varid{i\char95 2}}
  \end{lcbr}
%
\just\equiv{ Universal-+(17) }
%
  \begin{lcbr}
    \ensuremath{\Varid{outExpAr}\comp \underline{\Conid{X}}\mathrel{=}\Varid{i\char95 1}}\\
    \begin{lcbr}
      \ensuremath{\Varid{outExpAr}\comp \Conid{N}\mathrel{=}\Varid{i\char95 2}\comp \Varid{i\char95 1}}\\
      \ensuremath{\Varid{outExpAr}\comp [\mskip1.5mu \Varid{bin},\uncurry{\Conid{Un}}\mskip1.5mu]\mathrel{=}\Varid{i\char95 2}\comp \Varid{i\char95 2}}
    \end{lcbr}
  \end{lcbr}
%
\just\equiv{ Fusão-+(20) \& Universal-+(17) }
%
  \begin{lcbr}
    \ensuremath{\Varid{outExpAr}\comp \underline{\Conid{X}}\mathrel{=}\Varid{i\char95 1}}\\
    \begin{lcbr}
      \ensuremath{\Varid{outExpAr}\comp \Conid{N}\mathrel{=}\Varid{i\char95 2}\comp \Varid{i\char95 1}}\\
      \begin{lcbr}
        \ensuremath{\Varid{outExpAr}\comp \Varid{bin}\mathrel{=}\Varid{i\char95 2}\comp \Varid{i\char95 2}\comp \Varid{i\char95 1}}\\
        \ensuremath{\Varid{outExpAr}\comp \uncurry{\Conid{Un}}\mathrel{=}\Varid{i\char95 2}\comp \Varid{i\char95 2}\comp \Varid{i\char95 2}}
      \end{lcbr}
    \end{lcbr}
  \end{lcbr}  
%
\just\equiv{ Def-comp(72) \& Introdução de variáveis }
% 
  \begin{lcbr}
    \ensuremath{\Varid{outExpAr}\;\underline{\Conid{X}}\;()\mathrel{=}\Varid{i\char95 1}\;()}\\
    \begin{lcbr}
      \ensuremath{\Varid{outExpAr}\;(\Conid{N}\;(\Varid{a}))\mathrel{=}\Varid{i\char95 2}\;(\Varid{i\char95 1}\;(\Varid{a}))}\\
      \begin{lcbr}
        \ensuremath{\Varid{outExpAr}\;(\Varid{bin}\;(\Varid{op},(\Varid{a},\Varid{b})))\mathrel{=}\Varid{i\char95 2}\;(\Varid{i\char95 2}\;(\Varid{i\char95 1}\;(\Varid{op},(\Varid{a},\Varid{b}))))}\\
        \ensuremath{\Varid{outExpAr}\;\uncurry{\Conid{Un}}\;(\Varid{op},\Varid{a})\mathrel{=}\Varid{i\char95 2}\;(\Varid{i\char95 2}\;(\Varid{i\char95 2}\;(\Varid{op},\Varid{a})))}
      \end{lcbr}
    \end{lcbr}
  \end{lcbr}  
\qed
\end{eqnarray*}

\begin{eqnarray*}
\xymatrix@C=2cm{
    \ensuremath{\N_0}
&
    \ensuremath{\mathrm{1}\mathbin{+}\N_0}
           \ar[l]_-{\ensuremath{\mathsf{in}}}
}
\end{eqnarray*}

\begin{hscode}\SaveRestoreHook
\column{B}{@{}>{\hspre}l<{\hspost}@{}}%
\column{E}{@{}>{\hspre}l<{\hspost}@{}}%
\>[B]{}\Varid{outExpAr}\;\Conid{X}\mathrel{=}i_1\;(){}\<[E]%
\\
\>[B]{}\Varid{outExpAr}\;(\Conid{N}\;\Varid{a})\mathrel{=}i_2\;(i_1\;\Varid{a}){}\<[E]%
\\
\>[B]{}\Varid{outExpAr}\;(\Conid{Bin}\;\Varid{op}\;\Varid{a}\;\Varid{b})\mathrel{=}i_2\;(i_2\;(i_1\;(\Varid{op},(\Varid{a},\Varid{b})))){}\<[E]%
\\
\>[B]{}\Varid{outExpAr}\;(\Conid{Un}\;\Varid{op}\;\Varid{a})\mathrel{=}i_2\;(i_2\;(i_2\;(\Varid{op},\Varid{a}))){}\<[E]%
\ColumnHook
\end{hscode}\resethooks


Tanto no enunciado deste projeto como nas aulas os docentes iam-nos alertando para a importância de ler e analisar código. Na resolução deste exercício acabamos por perceber
essa importância, pois a análise do código do anexo \ref{sec:codigo} e/ou dos ficheiros da pasta \emph{src} mostraram-se essenciais na resolução dos problemas do projeto.
No caso desta segunda alínea do problema 1, depois de percebermos o funcionamento da função \ensuremath{\Varid{baseExpAr}} fornecida, o ficheiro \emph{List.hs} ajudou bastante na resolução deste
exercício.\\

Ao olharmos para a função \ensuremath{\Varid{baseExpAr}} percebemos que a mesma necessita de 7 argumentos. Acabamos por perceber que 7 argumentos eram estes ao interpretar o código da \emph{recList}.\\

Sendo o \emph{recList} o \emph{F f} das listas, fizemos a analogia para esta alínea, tendo em conta o \emph{outExpAr} definido na alínea anterior, uma vez que este gene só será executado depois dele.\\

Tendo em conta que \ensuremath{\Conid{N}\;\Varid{a}} representa um certo número e \ensuremath{\Conid{X}} uma variável, estes dois casos poderiam ser considerados como casos de paragem, uma vez que não são mais simplificáveis. 
Estas podem portanto ser do tipo \ensuremath{\Varid{id}}, fazendo aqui uma analogia clara com o caso de paragem das listas.  
Porém, nos outros 2 casos, a expressão pode necessitar de mais simplificações. No caso da \ensuremath{\Conid{Un}\;\Conid{UnOp}\;(\Conid{ExpAr}\;\Varid{a})} só a \ensuremath{\Conid{ExpAr}} precisa de ser simplificada, ficando com o tipo \ensuremath{\Varid{id}\times\Varid{f}}, tal como acontecia, aliás,
no caso das listas.
Já na expressão com o operador binário, é necessário simplificar mais uma expressão do que no caso anterior, ficando, portanto, com o tipo \ensuremath{\Varid{id}\times(\Varid{f}\times\Varid{f})}.\\

Relembrando a definição da função \ensuremath{\Varid{baseExpAr}}, que a a \ensuremath{\Varid{recExpAr}} invoca:\\

\ensuremath{\Varid{baseExpAr}\;\Varid{f}\;\Varid{g}\;\Varid{h}\;\Varid{j}\;\Varid{k}\;\Varid{l}\;\Varid{z}\mathrel{=}\Varid{f}\mathbin{+}(\Varid{g}\mathbin{+}(\Varid{h}\times(\Varid{j}\times\Varid{k})\mathbin{+}\Varid{l}\times\Varid{z}))}\\

podemos facilmente atribuir valores a \ensuremath{\Varid{f}},\ensuremath{\Varid{g}},\ensuremath{\Varid{h}},\ensuremath{\Varid{j}},\ensuremath{\Varid{k}},\ensuremath{\Varid{l}} e \ensuremath{\Varid{z}}:\\

\begin{itemize}
\item \ensuremath{\Varid{f}} é \ensuremath{\Varid{id}} e representa a \ensuremath{\Conid{X}}.
\item \ensuremath{\Varid{g}} é \ensuremath{\Varid{id}} e representa a \ensuremath{\Conid{N}\;\Varid{a}}.
\item \ensuremath{\Varid{h}\times(\Varid{j}\times\Varid{k})}  é \ensuremath{\Varid{id}\times(\Varid{f}\times\Varid{f})} e representa a \ensuremath{\Conid{Bin}\;\Conid{BinOp}\;(\Conid{ExpAr}\;\Varid{a})\;(\Conid{ExpAr}\;\Varid{a})}.\\
\ensuremath{\Varid{h}} é, portanto, \ensuremath{\Varid{id}}, \ensuremath{\Varid{j}} e \ensuremath{\Varid{k}} são \ensuremath{\Varid{f}}.
\item Por último, a expressão unária \ensuremath{\Conid{Un}\;\Conid{UnOp}\;(\Conid{ExpAr}\;\Varid{a})} é representada por \ensuremath{\Varid{l}\times\Varid{z}}, ou seja, \ensuremath{\Varid{id}\times\Varid{f}}.\\
\end{itemize}

\\A expressão final pretendida é \ensuremath{\Varid{id}\mathbin{+}(\Varid{id}\mathbin{+}(\Varid{id}\times(\Varid{f}\times\Varid{f})\mathbin{+}\Varid{id}\times\Varid{f}))}, que é facilmente inserida na função \ensuremath{\Varid{baseExpAr}} com as conclusões enumeradas em cima.
\\Temos, assim, reunidas as condições para afirmar que:\\

\begin{hscode}\SaveRestoreHook
\column{B}{@{}>{\hspre}l<{\hspost}@{}}%
\column{15}{@{}>{\hspre}l<{\hspost}@{}}%
\column{E}{@{}>{\hspre}l<{\hspost}@{}}%
\>[B]{}\Varid{recExpAr}\;\Varid{f}\mathrel{=}{}\<[15]%
\>[15]{}\Varid{baseExpAr}\;\Varid{id}\;\Varid{id}\;\Varid{id}\;\Varid{f}\;\Varid{f}\;\Varid{id}\;\Varid{f}{}\<[E]%
\ColumnHook
\end{hscode}\resethooks

\subsubsection{Alínea 2} \label{pg:P1.2}

Para a resolução desta alínea consideramos que o gene deveria simplificar expressões, aplicando as propriedades das expressões que as mesmas representam.\\
O gene deve ter, portanto, 2 argumentos: os mesmos 2 da \ensuremath{\Varid{eval\char95 exp}}. São estes: um \ensuremath{\Conid{Floating}\;\Varid{a}} e uma expressão aritmética.\\
Assim sendo, teremos 4 casos, estando os casos das operações binárias e unárias divididos em 2 subcasos:

\begin{hscode}\SaveRestoreHook
\column{B}{@{}>{\hspre}l<{\hspost}@{}}%
\column{3}{@{}>{\hspre}l<{\hspost}@{}}%
\column{25}{@{}>{\hspre}l<{\hspost}@{}}%
\column{E}{@{}>{\hspre}l<{\hspost}@{}}%
\>[B]{}\Varid{g\char95 eval\char95 exp}\;\Varid{x}\;(i_1\;()){}\<[25]%
\>[25]{}\mathrel{=}\Varid{x}{}\<[E]%
\\
\>[B]{}\Varid{g\char95 eval\char95 exp}\;\Varid{x}\;(i_2\;(i_1\;\Varid{a}))\mathrel{=}\Varid{a}{}\<[E]%
\\
\>[B]{}\Varid{g\char95 eval\char95 exp}\;\Varid{x}\;(i_2\;(i_2\;(i_1\;(\Varid{op},(\Varid{a},\Varid{b}))))){}\<[E]%
\\
\>[B]{}\hsindent{3}{}\<[3]%
\>[3]{}\mid \Varid{op}\equiv \Conid{Sum}\mathrel{=}\Varid{a}\mathbin{+}\Varid{b}{}\<[E]%
\\
\>[B]{}\hsindent{3}{}\<[3]%
\>[3]{}\mid \Varid{op}\equiv \Conid{Product}\mathrel{=}\Varid{a}\mathbin{*}\Varid{b}{}\<[E]%
\\
\>[B]{}\Varid{g\char95 eval\char95 exp}\;\Varid{x}\;(i_2\;(i_2\;(i_2\;(\Varid{op},\Varid{a})))){}\<[E]%
\\
\>[B]{}\hsindent{3}{}\<[3]%
\>[3]{}\mid \Varid{op}\equiv \Conid{Negate}\mathrel{=}\mathbin{-}\Varid{a}{}\<[E]%
\\
\>[B]{}\hsindent{3}{}\<[3]%
\>[3]{}\mid \Varid{op}\equiv \Conid{E}\mathrel{=}\Varid{expd}\;\Varid{a}{}\<[E]%
\ColumnHook
\end{hscode}\resethooks

\subsubsection{Alínea 3} \label{pg:P1.3}

A Simplificação das expressões é feita na função \ensuremath{\Varid{clean}}, cujo o nome sugestivo indicava que seria aqui que a expressão seria de facto simplificada.
Por outro lado, a função \ensuremath{\Varid{gopt}} poderia reaproveitar o código da alínea anterior(da função \ensuremath{\Varid{g\char95 eval\char95 exp}}), visto que esta função já tem definido o procedimento para simplificar expressões e só é corrida depois da \ensuremath{\Varid{clean}}.
Deste modo estariamos a efetuar cálculos somente se a \ensuremath{\Varid{clean}} não tivesse tirado já proveito dos elementos absorventes das operações.\\
E que elementos podem ser estes?\\
No nosso caso identificamos 2 bastante conhecidos das propriedades matemáticas:\\

\begin{itemize}
\item No caso da operação binária da multiplicação, sabe-se que que qualquer valor/expressão multiplicad por 0(no caso do exercício \ensuremath{\Conid{N}\;\mathrm{0}}) resulta em 0.\\
\item No caso da operação unária do exponencial de base \emph{e}, sabemos que se o número de Euler/Nepper estiver elevado a 0, ou seja no caso deste exercício, a \ensuremath{\Conid{N}\;\mathrm{0}},
terá como resultado o valor 1.
\end{itemize}


Temos, assim, reunidas as condições para afirmar que a \ensuremath{\Varid{clean}} de uma \ensuremath{\Conid{ExpAr}} pode ser dada por:\\

\begin{hscode}\SaveRestoreHook
\column{B}{@{}>{\hspre}l<{\hspost}@{}}%
\column{E}{@{}>{\hspre}l<{\hspost}@{}}%
\>[B]{}\Varid{clean}\;(\Conid{Bin}\;\Conid{Product}\;(\Conid{N}\;\mathrm{0})\;\Varid{b})\mathrel{=}i_2\;(i_1\;\mathrm{0}){}\<[E]%
\\
\>[B]{}\Varid{clean}\;(\Conid{Bin}\;\Conid{Product}\;\Varid{a}\;(\Conid{N}\;\mathrm{0}))\mathrel{=}i_2\;(i_1\;\mathrm{0}){}\<[E]%
\\
\>[B]{}\Varid{clean}\;(\Conid{Un}\;\Conid{E}\;(\Conid{N}\;\mathrm{0}))\mathrel{=}i_2\;(i_1\;\mathrm{1}){}\<[E]%
\\
\>[B]{}\Varid{clean}\;\Varid{exp}\mathrel{=}\Varid{outExpAr}\;\Varid{exp}{}\<[E]%
\ColumnHook
\end{hscode}\resethooks

A \ensuremath{\Varid{gopt}}, como foi referido, tira aproveitamento da definição da \ensuremath{\Varid{g\char95 eval\char95 exp}}:\\

\begin{hscode}\SaveRestoreHook
\column{B}{@{}>{\hspre}l<{\hspost}@{}}%
\column{E}{@{}>{\hspre}l<{\hspost}@{}}%
\>[B]{}\Varid{gopt}\;\Varid{x}\mathrel{=}\Varid{g\char95 eval\char95 exp}\;\Varid{x}{}\<[E]%
\ColumnHook
\end{hscode}\resethooks


\subsubsection{Alínea 4} \label{pg:P1.4}


Na resolução desta alínea, o mais importante foi perceber como é que, começando da expressão aritmética, poderiamos chegar ao par de \ensuremath{\Conid{ExpAr}}'s pretendido.\\

Ora, para tal é também essencial compreender que a regra da derivada do produto de 2 funções não só necessita da derivada de uma função como também necessita da 
função em si.\\

Analisando código, também isso uma parte essencial deste trabalho, podemos concluir que este gene verá o segundo elemento do par a ser selecionado pelo \ensuremath{\p2} na 
assinatura da função \ensuremath{\Varid{sd}}. Visto que o objetivo da função \ensuremath{\Varid{sd}} é calcular a derivada, podemos definir o seu gene como uma função que devolve o valor da função como primeiro
elemento do par e o valor da derivada no segundo elemento do par.

\begin{hscode}\SaveRestoreHook
\column{B}{@{}>{\hspre}l<{\hspost}@{}}%
\column{5}{@{}>{\hspre}l<{\hspost}@{}}%
\column{9}{@{}>{\hspre}l<{\hspost}@{}}%
\column{11}{@{}>{\hspre}l<{\hspost}@{}}%
\column{E}{@{}>{\hspre}l<{\hspost}@{}}%
\>[B]{}\Varid{sd\char95 gen}\mathbin{::}\Conid{Floating}\;\Varid{a}\Rightarrow {}\<[E]%
\\
\>[B]{}\hsindent{5}{}\<[5]%
\>[5]{}()+(\Varid{a}+((\Conid{BinOp},((\Conid{ExpAr}\;\Varid{a},\Conid{ExpAr}\;\Varid{a}),(\Conid{ExpAr}\;\Varid{a},\Conid{ExpAr}\;\Varid{a})))+(\Conid{UnOp},(\Conid{ExpAr}\;\Varid{a},\Conid{ExpAr}\;\Varid{a})))){}\<[E]%
\\
\>[B]{}\hsindent{5}{}\<[5]%
\>[5]{}\to (\Conid{ExpAr}\;\Varid{a},\Conid{ExpAr}\;\Varid{a}){}\<[E]%
\\
\>[B]{}\Varid{sd\char95 gen}\;{}\<[9]%
\>[9]{}(i_1\;())\mathrel{=}(\Conid{X},(\Conid{N}\;\mathrm{1})){}\<[E]%
\\
\>[B]{}\Varid{sd\char95 gen}\;{}\<[9]%
\>[9]{}(i_2\;(i_1\;\Varid{a}))\mathrel{=}((\Conid{N}\;\Varid{a}),(\Conid{N}\;\mathrm{0})){}\<[E]%
\\
\>[B]{}\Varid{sd\char95 gen}\;{}\<[9]%
\>[9]{}(i_2\;(i_2\;(i_1\;(\Conid{Sum},(\Varid{a},\Varid{b})))))\mathrel{=}(\Conid{Bin}\;\Conid{Sum}\;(\p1\;\Varid{a})\;(\p1\;\Varid{b}),\Conid{Bin}\;\Conid{Sum}\;(\p2\;\Varid{a})\;(\p2\;\Varid{b})){}\<[E]%
\\
\>[B]{}\Varid{sd\char95 gen}\;{}\<[9]%
\>[9]{}(i_2\;(i_2\;(i_1\;(\Conid{Product},(\Varid{a},\Varid{b})))))\mathrel{=}(\Conid{Bin}\;\Conid{Product}\;(\p1\;\Varid{a})\;(\p1\;\Varid{b}),\Conid{Bin}\;\Conid{Sum}\;\Varid{fst\char95 aux}\;\Varid{snd\char95 aux}){}\<[E]%
\\
\>[B]{}\hsindent{5}{}\<[5]%
\>[5]{}\mathbf{where}\;\Varid{fst\char95 aux}\mathrel{=}\Conid{Bin}\;\Conid{Product}\;(\p1\;\Varid{a})\;(\p2\;\Varid{b}){}\<[E]%
\\
\>[5]{}\hsindent{6}{}\<[11]%
\>[11]{}\Varid{snd\char95 aux}\mathrel{=}\Conid{Bin}\;\Conid{Product}\;(\p2\;\Varid{a})\;(\p1\;\Varid{b}){}\<[E]%
\\
\>[B]{}\Varid{sd\char95 gen}\;(i_2\;(i_2\;(i_2\;(\Conid{E},\Varid{a}))))\mathrel{=}(\Conid{Un}\;\Conid{E}\;(\p1\;\Varid{a}),\Conid{Bin}\;\Conid{Product}\;(\Conid{Un}\;\Conid{E}\;(\p1\;\Varid{a}))\;(\p2\;\Varid{a})){}\<[E]%
\\
\>[B]{}\Varid{sd\char95 gen}\;(i_2\;(i_2\;(i_2\;(\Conid{Negate},\Varid{a}))))\mathrel{=}(\Conid{Un}\;\Conid{Negate}\;(\p1\;\Varid{a}),\Conid{Un}\;\Conid{Negate}\;(\p2\;\Varid{a})){}\<[E]%
\ColumnHook
\end{hscode}\resethooks

\subsubsection{Alínea 5} \label{pg:P1.5}



\begin{hscode}\SaveRestoreHook
\column{B}{@{}>{\hspre}l<{\hspost}@{}}%
\column{E}{@{}>{\hspre}l<{\hspost}@{}}%
\>[B]{}\Varid{ad\char95 gen}\;\Varid{pnt}\;(i_1\;())\mathrel{=}(\Varid{pnt},\mathrm{1}){}\<[E]%
\\
\>[B]{}\Varid{ad\char95 gen}\;\Varid{pnt}\;(i_2\;(i_1\;\Varid{a}))\mathrel{=}(\Varid{a},\mathrm{0}){}\<[E]%
\\
\>[B]{}\Varid{ad\char95 gen}\;\Varid{pnt}\;(i_2\;(i_2\;(i_1\;(\Conid{Sum},(\Varid{a},\Varid{b})))))\mathrel{=}((\p1\;\Varid{a})\mathbin{+}(\p1\;\Varid{b}),(\p2\;\Varid{a})\mathbin{+}(\p2\;\Varid{b})){}\<[E]%
\\
\>[B]{}\Varid{ad\char95 gen}\;\Varid{pnt}\;(i_2\;(i_2\;(i_1\;(\Conid{Product},(\Varid{a},\Varid{b})))))\mathrel{=}((\p1\;\Varid{a})\mathbin{*}(\p1\;\Varid{b}),((\p1\;\Varid{a})\mathbin{*}(\p2\;\Varid{b}))\mathbin{+}((\p2\;\Varid{a})\mathbin{*}(\p1\;\Varid{b}))){}\<[E]%
\\
\>[B]{}\Varid{ad\char95 gen}\;\Varid{pnt}\;(i_2\;(i_2\;(i_2\;(\Conid{E},\Varid{a}))))\mathrel{=}(\Varid{expd}\;(\p1\;\Varid{a}),(\p2\;\Varid{a})\mathbin{*}\Varid{expd}\;((\p1\;\Varid{a}))){}\<[E]%
\\
\>[B]{}\Varid{ad\char95 gen}\;\Varid{pnt}\;(i_2\;(i_2\;(i_2\;(\Conid{Negate},\Varid{a}))))\mathrel{=}(\mathbin{-}(\p1\;\Varid{a}),\mathbin{-}(\p2\;\Varid{a})){}\<[E]%
\ColumnHook
\end{hscode}\resethooks

\subsection*{Problema 2}

Primeiramente, vamos tentar buscar uma relação entre o \emph{n}-ésimo valor de Catalon e o seu (\emph{n+1})-ésimo valor: \\
\begin{flalign}
        &\ \cfrac{C_{n+1}}{C_n} = \cfrac{\frac{(2(\emph{n}+1))!}{(\emph{n} + 2)!(\emph{n}+1)!}} {\frac{(2\emph{n})!}{(\emph{n}+1)!(\emph{n})!}}\notag\\
        \equiv&\notag\\
        &\ \cfrac{C_{n+1}}{C_n} = \frac{(2(n+1))!(\emph{n})!}{(\emph{n}+2)!(2\emph{n})!}\notag\\
        \equiv\notag\\  
        &\ \cfrac{C_{n+1}}{C_n} = \frac{(2\emph{n}+2)(2\emph{n}+1)(2\emph{n})!(\emph{n})!}{(\emph{n}+2)(\emph{n}+1)(\emph{n})!(2\emph{n})!}\notag\\
        \equiv\notag&\\
        &\ \cfrac{C_{n+1}}{C_n} = \frac{(2\emph{n}+2)(2\emph{n}+1)}{(\emph{n}+2)(\emph{n}+1)}\notag\\ 
        \equiv\notag&\\
        &\ \cfrac{C_{n+1}}{C_n} = \frac{2(\emph{n}+1)(2\emph{n}+1)}{(\emph{n}+2)(\emph{n}+1)}\notag\\
        \equiv\notag&\\
        &\ \cfrac{C_{n+1}}{C_n} = \frac{4\emph{n}+2}{\emph{n}+2}\notag\\
        \equiv\notag&\\
        &C_{n+1} = \frac{4\emph{n}+2}{\emph{n}+2} C_n\notag&
\end{flalign}

Tendo chegado a esta expressão, podemos agora dividir a equação entre 2 outras:\\

\begin{flalign}
&c(0) = 1&\notag\\
&c(n+1) = \frac{4\emph{n}+2}{\emph{n}+2} c(n)\notag&\\\notag\\\notag
&num(\emph{n}) = 4\emph{n} + 2&\notag\\
&num(0) = 2&\notag\\
&num(\emph{n+1}) = 4(\emph{n}+1) + 2 = 4 + num(\emph{n})\notag&\\\notag\\\notag
&den(\emph{n}) = \emph{n} + 2\notag&\\
&den(0) = 2&\notag\\
&den(\emph{n+1}) = (\emph{n}+3) = 1 + den(\emph{n})\notag&
\end{flalign}
\\
Pela \emph{regra da algibeira}, teremos:

\begin{hscode}\SaveRestoreHook
\column{B}{@{}>{\hspre}l<{\hspost}@{}}%
\column{6}{@{}>{\hspre}l<{\hspost}@{}}%
\column{E}{@{}>{\hspre}l<{\hspost}@{}}%
\>[B]{}\Varid{loop}\;(\Varid{a},\Varid{num},\Varid{den})\mathrel{=}(\Varid{a}\mathbin{*}\Varid{num}\mathbin{\Varid{`div`}}\Varid{den},\mathrm{4}\mathbin{+}\Varid{num},\mathrm{1}\mathbin{+}\Varid{den}){}\<[E]%
\\
\>[B]{}\Varid{inic}\mathrel{=}(\mathrm{1},\mathrm{2},\mathrm{2}){}\<[E]%
\\
\>[B]{}\Varid{prj}\;{}\<[6]%
\>[6]{}(\Varid{a},\Varid{b},\Varid{c})\mathrel{=}\Varid{a}{}\<[E]%
\ColumnHook
\end{hscode}\resethooks

por forma a que

\begin{hscode}\SaveRestoreHook
\column{B}{@{}>{\hspre}l<{\hspost}@{}}%
\column{E}{@{}>{\hspre}l<{\hspost}@{}}%
\>[B]{}\Varid{cat}\mathrel{=}\Varid{prj}\comp \for{\Varid{loop}}\ {\Varid{inic}}{}\<[E]%
\ColumnHook
\end{hscode}\resethooks
seja a função pretendida.

\subsection*{Problema 3}

\begin{hscode}\SaveRestoreHook
\column{B}{@{}>{\hspre}l<{\hspost}@{}}%
\column{5}{@{}>{\hspre}l<{\hspost}@{}}%
\column{E}{@{}>{\hspre}l<{\hspost}@{}}%
\>[B]{}\Varid{calcLine}\mathbin{::}\Conid{NPoint}\to (\Conid{NPoint}\to \Conid{OverTime}\;\Conid{NPoint}){}\<[E]%
\\
\>[B]{}\Varid{calcLine}\;\Varid{p}\;\Varid{q}\;\Varid{d}\mathrel{=}(\Varid{cataList}\;\Varid{h})\mathbin{\$}(\Varid{zip}\;\Varid{p}\;\Varid{q})\;\mathbf{where}{}\<[E]%
\\
\>[B]{}\hsindent{5}{}\<[5]%
\>[5]{}\Varid{h}\;(i_1\;())\mathrel{=}[\mskip1.5mu \mskip1.5mu]{}\<[E]%
\\
\>[B]{}\hsindent{5}{}\<[5]%
\>[5]{}\Varid{h}\;(i_2\;((\Varid{a1},\Varid{a2}),\Varid{t}))\mathrel{=}(\plus )\;(\Varid{singl}\mathbin{\$}(\Varid{linear1d}\;\Varid{a1}\;\Varid{a2}\;\Varid{d}))\;\Varid{t}{}\<[E]%
\ColumnHook
\end{hscode}\resethooks


\begin{hscode}\SaveRestoreHook
\column{B}{@{}>{\hspre}l<{\hspost}@{}}%
\column{7}{@{}>{\hspre}l<{\hspost}@{}}%
\column{11}{@{}>{\hspre}l<{\hspost}@{}}%
\column{21}{@{}>{\hspre}l<{\hspost}@{}}%
\column{25}{@{}>{\hspre}l<{\hspost}@{}}%
\column{E}{@{}>{\hspre}l<{\hspost}@{}}%
\>[B]{}\Varid{deCasteljau}\mathbin{::}[\mskip1.5mu \Conid{NPoint}\mskip1.5mu]\to \Conid{OverTime}\;\Conid{NPoint}{}\<[E]%
\\
\>[B]{}\Varid{deCasteljau}\;\Varid{list}\;\Varid{time}\mathrel{=}\Varid{hyloAlgForm}\;\Varid{alg}\;\Varid{coalg}\;\Varid{list}\;\mathbf{where}{}\<[E]%
\\
\>[B]{}\hsindent{7}{}\<[7]%
\>[7]{}\Varid{coalg}\;[\mskip1.5mu \mskip1.5mu]\mathrel{=}i_1\;[\mskip1.5mu \mskip1.5mu]{}\<[E]%
\\
\>[B]{}\hsindent{7}{}\<[7]%
\>[7]{}\Varid{coalg}\;[\mskip1.5mu \Varid{a}\mskip1.5mu]\mathrel{=}i_1\;\Varid{a}{}\<[E]%
\\
\>[B]{}\hsindent{7}{}\<[7]%
\>[7]{}\Varid{coalg}\;\Varid{list}\mathrel{=}i_2\;(\Varid{p},\Varid{q})\;\mathbf{where}{}\<[E]%
\\
\>[7]{}\hsindent{4}{}\<[11]%
\>[11]{}\Varid{p}\mathrel{=}\Varid{init}\;\Varid{list}{}\<[E]%
\\
\>[7]{}\hsindent{4}{}\<[11]%
\>[11]{}\Varid{q}\mathrel{=}\Varid{tail}\;\Varid{list}{}\<[E]%
\\
\>[B]{}\hsindent{7}{}\<[7]%
\>[7]{}\Varid{alg}\;(i_1\;\Varid{a}){}\<[21]%
\>[21]{}\mathrel{=}\Varid{a}{}\<[E]%
\\
\>[B]{}\hsindent{7}{}\<[7]%
\>[7]{}\Varid{alg}\;(i_2\;(\Varid{p},\Varid{q})){}\<[25]%
\>[25]{}\mathrel{=}\Varid{calcLine}\;\Varid{p}\;\Varid{q}\;\Varid{time}{}\<[E]%
\ColumnHook
\end{hscode}\resethooks


\begin{hscode}\SaveRestoreHook
\column{B}{@{}>{\hspre}l<{\hspost}@{}}%
\column{14}{@{}>{\hspre}l<{\hspost}@{}}%
\column{E}{@{}>{\hspre}l<{\hspost}@{}}%
\>[B]{}\Varid{hyloAlgForm}{}\<[14]%
\>[14]{}\mathrel{=}\Varid{hyloLTree}{}\<[E]%
\ColumnHook
\end{hscode}\resethooks


\subsection*{Problema 4}

Solução para listas não vazias:
\begin{hscode}\SaveRestoreHook
\column{B}{@{}>{\hspre}l<{\hspost}@{}}%
\column{E}{@{}>{\hspre}l<{\hspost}@{}}%
\>[B]{}\Varid{avg}\mathrel{=}\p1\comp \Varid{avg\char95 aux}{}\<[E]%
\ColumnHook
\end{hscode}\resethooks

Para usarmos um catamorfismo para listas não vazias vamos criar uma definição que não use listas vazias:


\begin{hscode}\SaveRestoreHook
\column{B}{@{}>{\hspre}l<{\hspost}@{}}%
\column{E}{@{}>{\hspre}l<{\hspost}@{}}%
\>[B]{}\Varid{out\char95 ex4}\;[\mskip1.5mu \Varid{a}\mskip1.5mu]\mathrel{=}i_1\;(\Varid{a}){}\<[E]%
\\
\>[B]{}\Varid{out\char95 ex4}\;(\Varid{h}\mathbin{:}\Varid{t})\mathrel{=}i_2\;(\Varid{h},\Varid{t}){}\<[E]%
\ColumnHook
\end{hscode}\resethooks



\begin{hscode}\SaveRestoreHook
\column{B}{@{}>{\hspre}l<{\hspost}@{}}%
\column{E}{@{}>{\hspre}l<{\hspost}@{}}%
\>[B]{}\Varid{in\char95 ex4}\mathrel{=}\alt{\Varid{singl}}{\Varid{cons}}{}\<[E]%
\ColumnHook
\end{hscode}\resethooks



\begin{hscode}\SaveRestoreHook
\column{B}{@{}>{\hspre}l<{\hspost}@{}}%
\column{E}{@{}>{\hspre}l<{\hspost}@{}}%
\>[B]{}\Varid{cataL\char95 ex4}\;\Varid{g}\mathrel{=}\Varid{g}\comp (\Varid{recList}\;(\Varid{cataL\char95 ex4}\;\Varid{g}))\comp \Varid{out\char95 ex4}{}\<[E]%
\ColumnHook
\end{hscode}\resethooks

\begin{eqnarray*}
\start
    \ensuremath{\Varid{avg\char95 aux}\mathrel{=}\cata{\alt{\Varid{b}}{\Varid{q}}}}
%
\just\equiv{ \ensuremath{\Varid{avg\char95 aux}\mathbin{:=}\conj{\Varid{avg}}{\length };\Varid{b}\mathrel{=}\conj{\Varid{b1}}{\Varid{b2}};\Varid{q}\mathrel{=}\conj{\Varid{q1}}{\Varid{q2}}} }
%
  \ensuremath{\Varid{aplit}\;\Varid{avg}\;\length \mathrel{=}\cata{\alt{\conj{\Varid{b1}}{\Varid{b2}}}{\conj{\Varid{q1}}{\Varid{q2}}}}}
%
\just\equiv{ Lei da troca }
%
  \ensuremath{\conj{\Varid{avg}}{\length }\mathrel{=}\cata{\conj{\alt{\Varid{b1}}{\Varid{q1}}}{\alt{\Varid{b2}}{\Varid{q2}}}}}
%
\just\equiv{ Fokkinga }
%
  \begin{lcbr}
    \ensuremath{\Varid{f}\comp \mathbf{in}\mathrel{=}\alt{\Varid{b1}}{\Varid{q1}}\comp \conj{\Varid{f}}{\Varid{g}}}\\
    \ensuremath{\Varid{g}\comp \mathbf{in}\mathrel{=}\alt{\Varid{b2}}{\Varid{q2}}\comp \conj{\Varid{f}}{\Varid{g}}}
  \end{lcbr}
%
\just\equiv{ \ensuremath{\mathbf{in}} := \ensuremath{\alt{\Varid{singl}}{\Varid{cons}}}; \ensuremath{\Conid{F}\;\Varid{f}} := \ensuremath{\Varid{id}\mathbin{+}\Varid{id}\times\Varid{f}} }
%
  \begin{lcbr}
    \ensuremath{\Varid{avg}\comp \alt{\Varid{singl}}{\Varid{cons}}\mathrel{=}\alt{\Varid{b1}}{\Varid{q1}}\comp (\Varid{id}\mathbin{+}\Varid{id}\times\conj{\Varid{avg}}{\length }}\\
    \ensuremath{\length \comp \alt{\Varid{singl}}{\Varid{cons}}\mathrel{=}\alt{\Varid{b2}}{\Varid{q2}}\comp (\Varid{id}\mathbin{+}\Varid{id}\times\conj{\Varid{avg}}{\length }}
  \end{lcbr}
%
\just\equiv{ Fusão - + ; Absorção - +; Natural -id}
%
  \begin{lcbr}
    \ensuremath{\alt{\Varid{avg}\comp \Varid{singl}}{\Varid{avg}\comp \Varid{cons}}\mathrel{=}\alt{\Varid{b1}}{\Varid{q1}\comp \conj{\p1}{\conj{\Varid{avg}\comp \p2}{\length \comp \p2}}}}\\
    \ensuremath{\alt{\length \comp \Varid{singl}}{\length \comp \Varid{cons}}\mathrel{=}\alt{\Varid{b2}}{\Varid{q2}\comp \conj{\p1}{\conj{\Varid{avg}\comp \p2}{\length \comp \p2}}}}
  \end{lcbr}
% 
\just\equiv{ Eq - + }
%
  \begin{lcbr}
    \begin{lcbr}
      \ensuremath{\Varid{avg}\comp \Varid{singl}\mathrel{=}\Varid{b1}} \\
      \ensuremath{\Varid{avg}\comp \Varid{cons}\mathrel{=}\Varid{q1}\comp \conj{\p1}{\conj{\Varid{avg}\comp \p2}{\length \comp \p2}}} 
    \end{lcbr} \\
    \begin{lcbr}
      \ensuremath{\length \comp \Varid{singl}\mathrel{=}\Varid{b2}}\\
      \ensuremath{\length \comp \Varid{cons}\mathrel{=}\Varid{q2}\comp \conj{\p1}{\conj{\Varid{avg}\comp \p2}{\length \comp \p2}}}
    \end{lcbr}
  \end{lcbr}  
%
\just\equiv{ Def-comp(72) ; Introdução de variáveis ; Def - x ; Def-split ; Def-singl ; Def-cons}
% 
  \begin{lcbr}
    \begin{lcbr}
      \ensuremath{\Varid{avg}\;[\mskip1.5mu \Varid{x}\mskip1.5mu]\mathrel{=}\Varid{b1}\;(\Varid{x},\Varid{xs})} \\
      \ensuremath{\Varid{avg}\;(\Varid{x}\mathbin{:}\Varid{xs})\mathrel{=}\Varid{q1}\;(\p1\;(\Varid{x},\Varid{xs}),(\Varid{avg}\;(\p2\;(\Varid{x},\Varid{xs})),(\length \;(\p2\;(\Varid{x},\Varid{xs})))))} 
    \end{lcbr} \\
    \begin{lcbr}
      \ensuremath{\length \;[\mskip1.5mu \Varid{x}\mskip1.5mu]\mathrel{=}\Varid{b2}\;(\Varid{x},\Varid{xs})}\\
      \ensuremath{\length \;(\Varid{x}\mathbin{:}\Varid{xs})\mathrel{=}\Varid{q2}\comp (\p1\;(\Varid{x},\Varid{xs}),(\Varid{avg}\;(\p2\;(\Varid{x},\Varid{xs})),(\length \;(\p2\;(\Varid{x},\Varid{xs})))))}
    \end{lcbr}
  \end{lcbr}
%
\just\equiv{ Def-proj ; avg[x] = x; length[x] = 1; length(x:xs) = succ . length(xs); Def-avg}
% 
  \begin{lcbr}
    \begin{lcbr}
      \ensuremath{\Varid{x}\mathrel{=}\Varid{b1}} \\
      \ensuremath{((\Varid{x}\mathbin{+}\length \mathbin{*}\Varid{avg}\;(\Varid{xs}))\mathbin{/}(\length \mathbin{+}\mathrm{1}))\mathrel{=}\Varid{q1}\;(\Varid{x},(\Varid{avg}\;(\Varid{xs}),(\length \;(\Varid{xs}))))} 
    \end{lcbr} \\
    \begin{lcbr}
      \ensuremath{\mathrm{1}\mathrel{=}\Varid{q1}}\\
      \ensuremath{\succ \comp \length \;(\Varid{xs})\mathrel{=}\Varid{q2}\;(\Varid{x},(\Varid{avg}\;(\Varid{xs}),(\length \;(\Varid{xs}))))}
    \end{lcbr}
  \end{lcbr}
%
\just\equiv{Simplificação}
%
  \begin{lcbr}
    \begin{lcbr}
      \ensuremath{\Varid{b1}\mathrel{=}\Varid{id}} \\
      \ensuremath{\Varid{q1}\;(\Varid{x},(\Varid{avg}\;(\Varid{xs}),(\length \;(\Varid{xs}))))\mathrel{=}((\Varid{x}\mathbin{+}\length \;(\Varid{xs})\mathbin{*}\Varid{avg}\;(\Varid{xs}))\mathbin{/}(\length \;(\Varid{xs})\mathbin{+}\mathrm{1}))}
    \end{lcbr} \\
    \begin{lcbr}
      \ensuremath{\Varid{b2}\mathrel{=}\mathrm{1}}\\
      \ensuremath{\Varid{q2}\mathrel{=}\succ \comp \p2\comp \p2}
    \end{lcbr}
  \end{lcbr}
\qed
\end{eqnarray*}

Substindo os valores na expressão inicial e para código Haskell temos:



\begin{hscode}\SaveRestoreHook
\column{B}{@{}>{\hspre}l<{\hspost}@{}}%
\column{7}{@{}>{\hspre}l<{\hspost}@{}}%
\column{E}{@{}>{\hspre}l<{\hspost}@{}}%
\>[B]{}\Varid{avg\char95 aux}\mathbin{::}[\mskip1.5mu \Conid{Double}\mskip1.5mu]\to (\Conid{Double},\Conid{Double}){}\<[E]%
\\
\>[B]{}\Varid{avg\char95 aux}\mathrel{=}\Varid{cataL\char95 ex4}\;\Varid{gene}\;\mathbf{where}{}\<[E]%
\\
\>[B]{}\hsindent{7}{}\<[7]%
\>[7]{}\Varid{gene}\mathrel{=}\alt{\conj{\Varid{id}}{\underline{\mathrm{1}}}}{\conj{\Varid{aux\char95 split}}{\succ \comp \p2\comp \p2}}{}\<[E]%
\\
\>[B]{}\hsindent{7}{}\<[7]%
\>[7]{}\Varid{aux\char95 split}\;(\Varid{h},(\Varid{a},\Varid{l}))\mathrel{=}(\Varid{h}\mathbin{+}(\Varid{l}\mathbin{*}\Varid{a}))\mathbin{/}(\Varid{l}\mathbin{+}\mathrm{1}){}\<[E]%
\ColumnHook
\end{hscode}\resethooks



Solução para árvores de tipo \LTree:

\begin{eqnarray*}
\start
    \ensuremath{\Varid{avg\char95 aux}\mathrel{=}\cata{\alt{\Varid{b}}{\Varid{q}}}}
%
\just\equiv{ \ensuremath{\Varid{avg\char95 aux}\mathbin{:=}\conj{\Varid{avg}}{\length };\Varid{b}\mathrel{=}\conj{\Varid{b1}}{\Varid{b2}};\Varid{q}\mathrel{=}\conj{\Varid{q1}}{\Varid{q2}}} }
%
  \ensuremath{\Varid{aplit}\;\Varid{avg}\;\length \mathrel{=}\cata{\alt{\conj{\Varid{b1}}{\Varid{b2}}}{\conj{\Varid{q1}}{\Varid{q2}}}}}
%
\just\equiv{ Lei da troca }
%
  \ensuremath{\conj{\Varid{avg}}{\length }\mathrel{=}\cata{\conj{\alt{\Varid{b1}}{\Varid{q1}}}{\alt{\Varid{b2}}{\Varid{q2}}}}}
%
\just\equiv{ Fokkinga }
%
  \begin{lcbr}
    \ensuremath{\Varid{f}\comp \mathbf{in}\mathrel{=}\alt{\Varid{b1}}{\Varid{q1}}\comp \conj{\Varid{f}}{\Varid{g}}}\\
    \ensuremath{\Varid{g}\comp \mathbf{in}\mathrel{=}\alt{\Varid{b2}}{\Varid{q2}}\comp \conj{\Varid{f}}{\Varid{g}}}
  \end{lcbr}
%
\just\equiv{ \ensuremath{\mathbf{in}} := \ensuremath{\alt{\Conid{Leaf}}{\Conid{Fork}}}; \ensuremath{\Conid{F}\;\Varid{f}} := id + f² }
%
  \begin{lcbr}
    \ensuremath{\Varid{avg}\comp \alt{\Conid{Leaf}}{\Conid{Fork}}\mathrel{=}\alt{\Varid{b1}}{\Varid{q1}}\comp (\Varid{id}\mathbin{+}\conj{\Varid{avg}}{\length }\times\conj{\Varid{avg}}{\length })}\\
    \ensuremath{\length \comp \alt{\Conid{Leaf}}{\Conid{Fork}}\mathrel{=}\alt{\Varid{b2}}{\Varid{q2}}\comp (\Varid{id}\mathbin{+}\conj{\Varid{avg}}{\length }\times\conj{\Varid{avg}}{\length })}
  \end{lcbr}
%
\just\equiv{ Fusão - + ; Absorção - +; Natural -id; Eq - +}
%
  \begin{lcbr}
    \begin{lcbr}
      \ensuremath{\Varid{avg}\comp \Conid{Leaf}\mathrel{=}\Varid{b1}} \\
      \ensuremath{\Varid{avg}\comp \Conid{Fork}\mathrel{=}\Varid{q1}\comp (\conj{\Varid{avg}}{\length }\times\conj{\Varid{avg}}{\length })} 
    \end{lcbr} \\
    \begin{lcbr}
      \ensuremath{\length \comp \Conid{Leaf}\mathrel{=}\Varid{b2}}\\
      \ensuremath{\length \comp \Conid{Fork}\mathrel{=}\Varid{q2}\comp (\conj{\Varid{avg}}{\length }\times\conj{\Varid{avg}}{\length })}
    \end{lcbr}
  \end{lcbr}  
% 
\just\equiv{ Introdução de variáveis; Def-comp; Def-x; Def-split}
%
  \begin{lcbr}
    \begin{lcbr}
      \ensuremath{\Varid{avg}\;(\Conid{Leaf}\;(\Varid{x}))\mathrel{=}\Varid{b1}\;(\Varid{x})} \\
      \ensuremath{\Varid{avg}\;(\Conid{Fork}\;((\Varid{x},\Varid{xs}),(\Varid{y},\Varid{ys})))\mathrel{=}\Varid{q1}\;((\Varid{avg}\;(\Varid{x},\Varid{xs}),\length \;(\Varid{x},\Varid{xs})),(\Varid{avg}\;(\Varid{y},\Varid{ys}),\length \;(\Varid{y},\Varid{ys})))} 
    \end{lcbr} \\
    \begin{lcbr}
      \ensuremath{\length \;(\Conid{Leaf}\;(\Varid{x}))\mathrel{=}\Varid{b2}\;(\Varid{x})} \\
      \ensuremath{\length \;(\Conid{Fork}\;((\Varid{x},\Varid{xs}),(\Varid{y},\Varid{ys})))\mathrel{=}\Varid{q2}\;((\Varid{avg}\;(\Varid{x},\Varid{xs}),\length \;(\Varid{x},\Varid{xs})),(\Varid{avg}\;(\Varid{y},\Varid{ys}),\length \;(\Varid{y},\Varid{ys})))} 
    \end{lcbr}
  \end{lcbr}  
%
\just\equiv{Def-avg; Def-length; Simplificação}
% 
  \begin{lcbr}
    \begin{lcbr}
      \ensuremath{\Varid{b1}\mathrel{=}\Varid{x}} \\
      \ensuremath{\Varid{q1}\;((\Varid{a1},\Varid{a2}),(\Varid{b1},\Varid{b2}))\mathrel{=}(\Varid{a1}\mathbin{*}\Varid{a2}\mathbin{+}\Varid{b1}\mathbin{*}\Varid{b2})\mathbin{/}(\Varid{a2}\mathbin{+}\Varid{b2})} 
    \end{lcbr} \\
    \begin{lcbr}
      \ensuremath{\Varid{b2}\mathrel{=}\mathrm{1}} \\
      \ensuremath{\Varid{q2}\mathrel{=}\uncurry{(\mathbin{+})}\comp \conj{\p2\comp \p1}{\p2\comp \p2}} 
    \end{lcbr}
  \end{lcbr}  
%
\end{eqnarray*}

Substindo os valores na expressão inicial e para código Haskell temos:


\begin{hscode}\SaveRestoreHook
\column{B}{@{}>{\hspre}l<{\hspost}@{}}%
\column{5}{@{}>{\hspre}l<{\hspost}@{}}%
\column{E}{@{}>{\hspre}l<{\hspost}@{}}%
\>[B]{}\Varid{avgLTree}\mathbin{::}\mathsf{LTree}\;\Conid{Double}\to \Conid{Double}{}\<[E]%
\\
\>[B]{}\Varid{avgLTree}\mathrel{=}\p1\comp \llparenthesis\, \Varid{gene}\,\rrparenthesis{}\<[E]%
\\
\>[B]{}\Varid{gene}\mathrel{=}\alt{\conj{\Varid{id}}{\underline{\mathrm{1}}}}{\conj{\Varid{aux\char95 split}}{\Varid{f}}}\;\mathbf{where}{}\<[E]%
\\
\>[B]{}\hsindent{5}{}\<[5]%
\>[5]{}\Varid{f}\mathrel{=}\uncurry{(\mathbin{+})}\comp \conj{\p2\comp \p2}{\p2\comp \p1}{}\<[E]%
\\
\>[B]{}\hsindent{5}{}\<[5]%
\>[5]{}\Varid{aux\char95 split}\;((\Varid{a1},\Varid{l1}),(\Varid{a2},\Varid{l2}))\mathrel{=}(\Varid{a1}\mathbin{*}\Varid{l1}\mathbin{+}\Varid{a2}\mathbin{*}\Varid{l2})\mathbin{/}(\Varid{l1}\mathbin{+}\Varid{l2}){}\<[E]%
\ColumnHook
\end{hscode}\resethooks




\subsection*{Problema 5}
Inserir em baixo o código \Fsharp\ desenvolvido, entre \text{\ttfamily \char92{}begin\char123{}verbatim\char125{}} e \text{\ttfamily \char92{}end\char123{}verbatim\char125{}}:

\begin{tabbing}\ttfamily
~module~BTree\\
\ttfamily ~\\
\ttfamily ~open~Cp\\
\ttfamily ~\\
\ttfamily ~\char47{}\char47{}~\char40{}1\char41{}~Datatype~definition~\char45{}\char45{}\char45{}\char45{}\char45{}\char45{}\char45{}\char45{}\char45{}\char45{}\char45{}\char45{}\char45{}\char45{}\char45{}\char45{}\char45{}\char45{}\char45{}\char45{}\char45{}\char45{}\char45{}\char45{}\char45{}\char45{}\char45{}\char45{}\char45{}\char45{}\char45{}\char45{}\char45{}\char45{}\char45{}\char45{}\char45{}\char45{}\char45{}\char45{}\char45{}\char45{}\char45{}\char45{}\char45{}\char45{}\char45{}\char45{}\char45{}\char45{}\char45{}\char45{}\char45{}\\
\ttfamily ~\\
\ttfamily ~type~BTree\char60{}\char39{}a\char62{}~\char61{}~Empty~\char124{}~Node~of~\char39{}a~\char42{}~\char40{}BTree\char60{}\char39{}a\char62{}~\char42{}~BTree\char60{}\char39{}a\char62{}\char41{}\\
\ttfamily ~\\
\ttfamily ~let~inBTree~x~\char61{}~either~\char40{}konst~Empty\char41{}~Node~x\\
\ttfamily ~\\
\ttfamily ~let~outBTree~x~\char61{}\\
\ttfamily ~~~~~match~x~with\\
\ttfamily ~~~~~\char124{}~Empty~\char45{}\char62{}~i1\char40{}\char41{}\\
\ttfamily ~~~~~\char124{}~Node~\char40{}a\char44{}\char40{}t1\char44{}t2\char41{}\char41{}~\char45{}\char62{}~i2~\char40{}a\char44{}\char40{}t1\char44{}t2\char41{}\char41{}\\
\ttfamily ~\\
\ttfamily ~\char47{}\char47{}~\char40{}2\char41{}~Ana~\char43{}~cata~\char43{}~hylo~\char45{}\char45{}\char45{}\char45{}\char45{}\char45{}\char45{}\char45{}\char45{}\char45{}\char45{}\char45{}\char45{}\char45{}\char45{}\char45{}\char45{}\char45{}\char45{}\char45{}\char45{}\char45{}\char45{}\char45{}\char45{}\char45{}\char45{}\char45{}\char45{}\char45{}\char45{}\char45{}\char45{}\char45{}\char45{}\char45{}\char45{}\char45{}\char45{}\char45{}\char45{}\char45{}\char45{}\char45{}\char45{}\char45{}\char45{}\char45{}\char45{}\char45{}\char45{}\char45{}\char45{}\char45{}\char45{}\\
\ttfamily ~\\
\ttfamily ~let~baseBTree~f~g~\char61{}~id~\char45{}\char124{}\char45{}~\char40{}f~\char62{}\char60{}~\char40{}g~\char62{}\char60{}~g\char41{}\char41{}\\
\ttfamily ~\\
\ttfamily ~let~recBTree~g~\char61{}~baseBTree~id~g\\
\ttfamily ~\\
\ttfamily ~let~rec~cataBTree~g~\char61{}~g~\char60{}\char60{}~\char40{}recBTree~\char40{}cataBTree~g\char41{}\char41{}~\char60{}\char60{}~outBTree\\
\ttfamily ~\\
\ttfamily ~let~rec~anaBTree~g~\char61{}~inBTree~\char60{}\char60{}~\char40{}recBTree~\char40{}anaBTree~g\char41{}~\char41{}~\char60{}\char60{}~g\\
\ttfamily ~\\
\ttfamily ~let~hyloBTree~h~g~\char61{}~cataBTree~h~\char60{}\char60{}~anaBTree~g\\
\ttfamily ~\\
\ttfamily ~\char47{}\char47{}~\char40{}3\char41{}~Map~\char45{}\char45{}\char45{}\char45{}\char45{}\char45{}\char45{}\char45{}\char45{}\char45{}\char45{}\char45{}\char45{}\char45{}\char45{}\char45{}\char45{}\char45{}\char45{}\char45{}\char45{}\char45{}\char45{}\char45{}\char45{}\char45{}\char45{}\char45{}\char45{}\char45{}\char45{}\char45{}\char45{}\char45{}\char45{}\char45{}\char45{}\char45{}\char45{}\char45{}\char45{}\char45{}\char45{}\char45{}\char45{}\char45{}\char45{}\char45{}\char45{}\char45{}\char45{}\char45{}\char45{}\char45{}\char45{}\char45{}\char45{}\char45{}\char45{}\char45{}\char45{}\char45{}\char45{}\char45{}\char45{}\char45{}\char45{}\char45{}\char45{}\\
\ttfamily ~\\
\ttfamily ~\char47{}\char47{}instance~Functor~BTree\\
\ttfamily ~\char47{}\char47{}~~~~~~~~~where~fmap~f~\char61{}~cataBTree~\char40{}~inBTree~\char46{}~baseBTree~f~id~\char41{}\\
\ttfamily ~let~fmap~f~\char61{}~cataBTree~\char40{}~inBTree~\char60{}\char60{}~baseBTree~f~id~\char41{}\\
\ttfamily ~\\
\ttfamily ~\char47{}\char47{}~\char40{}4\char41{}~Examples~\char45{}\char45{}\char45{}\char45{}\char45{}\char45{}\char45{}\char45{}\char45{}\char45{}\char45{}\char45{}\char45{}\char45{}\char45{}\char45{}\char45{}\char45{}\char45{}\char45{}\char45{}\char45{}\char45{}\char45{}\char45{}\char45{}\char45{}\char45{}\char45{}\char45{}\char45{}\char45{}\char45{}\char45{}\char45{}\char45{}\char45{}\char45{}\char45{}\char45{}\char45{}\char45{}\char45{}\char45{}\char45{}\char45{}\char45{}\char45{}\char45{}\char45{}\char45{}\char45{}\char45{}\char45{}\char45{}\char45{}\char45{}\char45{}\char45{}\char45{}\char45{}\char45{}\char45{}\char45{}\\
\ttfamily ~\\
\ttfamily ~\char47{}\char47{}~\char40{}4\char46{}0\char41{}~Inversion~\char40{}mirror\char41{}~\char45{}\char45{}\char45{}\char45{}\char45{}\char45{}\char45{}\char45{}\char45{}\char45{}\char45{}\char45{}\char45{}\char45{}\char45{}\char45{}\char45{}\char45{}\char45{}\char45{}\char45{}\char45{}\char45{}\char45{}\char45{}\char45{}\char45{}\char45{}\char45{}\char45{}\char45{}\char45{}\char45{}\char45{}\char45{}\char45{}\char45{}\char45{}\char45{}\char45{}\char45{}\char45{}\char45{}\char45{}\char45{}\char45{}\char45{}\char45{}\char45{}\char45{}\char45{}\char45{}\\
\ttfamily ~\\
\ttfamily ~let~invBTree~x~\char61{}~cataBTree~\char40{}inBTree~\char60{}\char60{}~\char40{}id~\char45{}\char124{}\char45{}~\char40{}id~\char62{}\char60{}~swap\char41{}\char41{}\char41{}~x\\
\ttfamily ~\\
\ttfamily ~\char47{}\char47{}~\char40{}4\char46{}2\char41{}~Counting~\char45{}\char45{}\char45{}\char45{}\char45{}\char45{}\char45{}\char45{}\char45{}\char45{}\char45{}\char45{}\char45{}\char45{}\char45{}\char45{}\char45{}\char45{}\char45{}\char45{}\char45{}\char45{}\char45{}\char45{}\char45{}\char45{}\char45{}\char45{}\char45{}\char45{}\char45{}\char45{}\char45{}\char45{}\char45{}\char45{}\char45{}\char45{}\char45{}\char45{}\char45{}\char45{}\char45{}\char45{}\char45{}\char45{}\char45{}\char45{}\char45{}\char45{}\char45{}\char45{}\char45{}\char45{}\char45{}\char45{}\char45{}\char45{}\char45{}\char45{}\char45{}\char45{}\\
\ttfamily ~\\
\ttfamily ~let~countBTree~x~\char61{}~cataBTree~\char40{}either~\char40{}konst~0\char41{}~\char40{}succ~\char60{}\char60{}~\char40{}uncurry~\char40{}\char43{}\char41{}\char41{}~\char60{}\char60{}~p2\char41{}\char41{}~x\\
\ttfamily ~\\
\ttfamily ~\char47{}\char47{}~\char40{}4\char46{}3\char41{}~Serialization~\char45{}\char45{}\char45{}\char45{}\char45{}\char45{}\char45{}\char45{}\char45{}\char45{}\char45{}\char45{}\char45{}\char45{}\char45{}\char45{}\char45{}\char45{}\char45{}\char45{}\char45{}\char45{}\char45{}\char45{}\char45{}\char45{}\char45{}\char45{}\char45{}\char45{}\char45{}\char45{}\char45{}\char45{}\char45{}\char45{}\char45{}\char45{}\char45{}\char45{}\char45{}\char45{}\char45{}\char45{}\char45{}\char45{}\char45{}\char45{}\char45{}\char45{}\char45{}\char45{}\char45{}\char45{}\char45{}\char45{}\char45{}\\
\ttfamily ~\\
\ttfamily ~let~inord~x~\char61{}\\
\ttfamily ~~~~~let~join~\char40{}x~\char44{}~\char40{}l~\char44{}~r\char41{}\char41{}~\char61{}~l~\char64{}~\char91{}x\char93{}~\char64{}~r\\
\ttfamily ~~~~~in~\char40{}either~nil~join\char41{}~x\\
\ttfamily ~\\
\ttfamily ~let~inordt~x~\char61{}~cataBTree~inord~x\\
\ttfamily ~\\
\ttfamily ~let~preord~x~\char61{}~\\
\ttfamily ~~~~~let~f~\char40{}x~\char44{}~\char40{}l~\char44{}~r\char41{}\char41{}~\char61{}~x~\char58{}\char58{}~l~\char64{}~r\\
\ttfamily ~~~~~in~\char40{}either~nil~f\char41{}~x\\
\ttfamily ~\\
\ttfamily ~let~preordt~x~\char61{}~cataBTree~preord~x~\char47{}\char47{}~pre\char45{}order~traversal\\
\ttfamily ~~~~~\\
\ttfamily ~let~postordt~x~\char61{}\\
\ttfamily ~~~~~let~f~\char40{}x~\char44{}~\char40{}l~\char44{}~r\char41{}\char41{}~\char61{}~l~\char64{}~r~\char64{}~\char91{}x\char93{}\\
\ttfamily ~~~~~in~cataBTree~\char40{}either~nil~f\char41{}~x\\
\ttfamily ~\\
\ttfamily ~\char47{}\char47{}~\char40{}4\char46{}4\char41{}~Quicksort~\char45{}\char45{}\char45{}\char45{}\char45{}\char45{}\char45{}\char45{}\char45{}\char45{}\char45{}\char45{}\char45{}\char45{}\char45{}\char45{}\char45{}\char45{}\char45{}\char45{}\char45{}\char45{}\char45{}\char45{}\char45{}\char45{}\char45{}\char45{}\char45{}\char45{}\char45{}\char45{}\char45{}\char45{}\char45{}\char45{}\char45{}\char45{}\char45{}\char45{}\char45{}\char45{}\char45{}\char45{}\char45{}\char45{}\char45{}\char45{}\char45{}\char45{}\char45{}\char45{}\char45{}\char45{}\char45{}\char45{}\char45{}\char45{}\char45{}\char45{}\char45{}\\
\ttfamily ~\\
\ttfamily ~let~rec~part~p~x~\char61{}\\
\ttfamily ~~~~~match~x~with\\
\ttfamily ~~~~~\char124{}~\char91{}\char93{}~\char45{}\char62{}~\char40{}\char91{}\char93{}\char44{}\char91{}\char93{}\char41{}\\
\ttfamily ~~~~~\char124{}~\char40{}h\char58{}\char58{}t\char41{}~\char45{}\char62{}~if~\char40{}p~h\char41{}~then~let~\char40{}s\char44{}l\char41{}~\char61{}~part~p~t~in~\char40{}\char40{}h\char58{}\char58{}s\char41{}\char44{}l\char41{}~else~let~\char40{}s\char44{}l\char41{}~\char61{}~part~p~t~in~\char40{}s\char44{}\char40{}h\char58{}\char58{}l\char41{}\char41{}\\
\ttfamily ~\\
\ttfamily ~let~less~h~x~\char61{}~\char40{}if~\char40{}x~\char60{}~h\char41{}~then~true~else~false\char41{}\\
\ttfamily ~\\
\ttfamily ~let~qsep~x~\char61{}\\
\ttfamily ~~~~~match~x~with\\
\ttfamily ~~~~~\char124{}~\char91{}\char93{}~\char45{}\char62{}~i1\char40{}\char41{}\\
\ttfamily ~~~~~\char124{}~\char40{}h\char58{}\char58{}t\char41{}~\char45{}\char62{}~let~\char40{}s\char44{}l\char41{}~\char61{}~part~\char40{}less~h\char41{}~t~in~i2~\char40{}h\char44{}\char40{}s\char44{}l\char41{}\char41{}\\
\ttfamily ~\\
\ttfamily ~let~qSort~x~\char61{}~hyloBTree~inord~qsep~x\\
\ttfamily ~\\
\ttfamily ~\char47{}\char47{}~\char40{}4\char46{}5\char41{}~Traces~\char45{}\char45{}\char45{}\char45{}\char45{}\char45{}\char45{}\char45{}\char45{}\char45{}\char45{}\char45{}\char45{}\char45{}\char45{}\char45{}\char45{}\char45{}\char45{}\char45{}\char45{}\char45{}\char45{}\char45{}\char45{}\char45{}\char45{}\char45{}\char45{}\char45{}\char45{}\char45{}\char45{}\char45{}\char45{}\char45{}\char45{}\char45{}\char45{}\char45{}\char45{}\char45{}\char45{}\char45{}\char45{}\char45{}\char45{}\char45{}\char45{}\char45{}\char45{}\char45{}\char45{}\char45{}\char45{}\char45{}\char45{}\char45{}\char45{}\char45{}\char45{}\char45{}\char45{}\char45{}\\
\ttfamily ~\\
\ttfamily ~let~rec~init~x~\char61{}\\
\ttfamily ~~~~~match~x~with\\
\ttfamily ~~~~~\char124{}~\char91{}\char93{}~\char45{}\char62{}~\char91{}\char93{}\\
\ttfamily ~~~~~\char124{}~\char40{}h\char58{}\char58{}t\char41{}~\char45{}\char62{}~\char91{}h\char93{}~\char64{}~\char40{}init~t\char41{}\\
\ttfamily ~\\
\ttfamily ~let~rec~last~x~\char61{}\\
\ttfamily ~~~~~match~x~with\\
\ttfamily ~~~~~\char124{}~\char91{}a\char93{}~\char45{}\char62{}~a\\
\ttfamily ~~~~~\char124{}~\char40{}h\char58{}\char58{}t\char41{}~\char45{}\char62{}~last~t\\
\ttfamily ~\\
\ttfamily ~let~rec~isOnList~x~\char61{}\\
\ttfamily ~~~~~match~x~with\\
\ttfamily ~~~~~\char124{}~\char40{}b~\char44{}~\char91{}\char93{}\char41{}~\char45{}\char62{}~false\\
\ttfamily ~~~~~\char124{}~\char40{}b~\char44{}~\char40{}h\char58{}\char58{}t\char41{}\char41{}~\char45{}\char62{}~if~\char40{}b~\char61{}~h\char41{}~then~true~else~isOnList~\char40{}b~\char44{}~t\char41{}\\
\ttfamily ~\\
\ttfamily ~let~rec~union~x~\char61{}\\
\ttfamily ~~~~~match~x~with\\
\ttfamily ~~~~~\char124{}~\char40{}\char91{}\char93{}~\char44{}~a\char41{}~\char45{}\char62{}~a\\
\ttfamily ~~~~~\char124{}~\char40{}a~\char44{}~\char91{}\char93{}\char41{}~\char45{}\char62{}~a\\
\ttfamily ~~~~~\char124{}~\char40{}a~\char44{}~b\char41{}~\char45{}\char62{}~if~\char40{}isOnList~\char40{}\char40{}last~b\char41{}~\char44{}~a\char41{}\char41{}~then~\char40{}union~\char40{}a~\char44{}~\char40{}init~b\char41{}\char41{}\char41{}~else~\char40{}\char40{}union~\char40{}a~\char44{}~\char40{}init~b\char41{}\char41{}\char41{}~\char64{}~\char91{}\char40{}last~b\char41{}\char93{}\char41{}\\
\ttfamily ~\\
\ttfamily ~let~headbtl~a~l~\char61{}~\char40{}a\char58{}\char58{}l\char41{}\\
\ttfamily ~\\
\ttfamily ~let~tunion~\char40{}a\char44{}\char40{}l\char44{}r\char41{}\char41{}~\char61{}~union~\char40{}\char40{}List\char46{}map~\char40{}headbtl~a\char41{}~l\char41{}~\char44{}~\char40{}List\char46{}map~\char40{}headbtl~a\char41{}~r\char41{}\char41{}\\
\ttfamily ~\\
\ttfamily ~let~traces~x~\char61{}~cataBTree~\char40{}either~\char40{}konst~\char91{}\char91{}\char93{}\char93{}\char41{}~tunion\char41{}~x\\
\ttfamily ~~\\
\ttfamily ~\char47{}\char47{}~\char40{}4\char46{}6\char41{}~Towers~of~Hanoi~\char45{}\char45{}\char45{}\char45{}\char45{}\char45{}\char45{}\char45{}\char45{}\char45{}\char45{}\char45{}\char45{}\char45{}\char45{}\char45{}\char45{}\char45{}\char45{}\char45{}\char45{}\char45{}\char45{}\char45{}\char45{}\char45{}\char45{}\char45{}\char45{}\char45{}\char45{}\char45{}\char45{}\char45{}\char45{}\char45{}\char45{}\char45{}\char45{}\char45{}\char45{}\char45{}\char45{}\char45{}\char45{}\char45{}\char45{}\char45{}\char45{}\char45{}\char45{}\char45{}\char45{}\char45{}\char45{}\\
\ttfamily ~\\
\ttfamily ~\char47{}\char47{}~pointwise\char58{}\\
\ttfamily ~\char47{}\char47{}~hanoi\char40{}d\char44{}0\char41{}~\char61{}~\char91{}\char93{}\\
\ttfamily ~\char47{}\char47{}~hanoi\char40{}d\char44{}n\char43{}1\char41{}~\char61{}~\char40{}hanoi~\char40{}not~d\char44{}n\char41{}\char41{}~\char43{}\char43{}~\char91{}\char40{}n\char44{}d\char41{}\char93{}~\char43{}\char43{}~\char40{}hanoi~\char40{}not~d\char44{}~n\char41{}\char41{}\\
\ttfamily ~\\
\ttfamily ~let~strategy~x~\char61{}\\
\ttfamily ~~~~~match~x~with\\
\ttfamily ~~~~~\char124{}~\char40{}d\char44{}0\char41{}~\char45{}\char62{}~i1~\char40{}\char41{}\\
\ttfamily ~~~~~\char124{}~\char40{}d\char44{}n\char41{}~\char45{}\char62{}~i2~\char40{}\char40{}n\char44{}d\char41{}\char44{}\char40{}\char40{}not~d\char44{}\char40{}n\char45{}1\char41{}\char41{}\char44{}\char40{}not~d\char44{}\char40{}n\char45{}1\char41{}\char41{}\char41{}\char41{}\\
\ttfamily ~\\
\ttfamily ~let~present~x~\char61{}~inord~x\\
\ttfamily ~\\
\ttfamily ~let~hanoi~x~\char61{}~hyloBTree~present~strategy~x\\
\ttfamily ~\\
\ttfamily ~\char47{}\char47{}~\char40{}5\char41{}~Depth~and~balancing~\char40{}using~mutual~recursion\char41{}~\char45{}\char45{}\char45{}\char45{}\char45{}\char45{}\char45{}\char45{}\char45{}\char45{}\char45{}\char45{}\char45{}\char45{}\char45{}\char45{}\char45{}\char45{}\char45{}\char45{}\char45{}\char45{}\char45{}\char45{}\char45{}\char45{}\\
\ttfamily ~\\
\ttfamily ~let~f\char40{}\char40{}b1\char44{}d1\char41{}\char44{}\char40{}b2\char44{}d2\char41{}\char41{}~\char61{}~\char40{}\char40{}b1\char44{}b2\char41{}\char44{}\char40{}d1\char44{}d2\char41{}\char41{}\\
\ttfamily ~\\
\ttfamily ~let~h\char40{}a\char44{}\char40{}\char40{}b1\char44{}b2\char41{}\char44{}\char40{}d1\char44{}d2\char41{}\char41{}\char41{}~\char61{}~\char40{}b1~\char38{}\char38{}~b2~\char38{}\char38{}~abs\char40{}d1\char45{}d2\char41{}\char60{}\char61{}1\char44{}1\char43{}max~d1~d2\char41{}\\
\ttfamily ~\\
\ttfamily ~let~baldepth~x~\char61{}~\\
\ttfamily ~~~~~let~g~x~\char61{}~either~\char40{}konst\char40{}true\char44{}1\char41{}\char41{}~\char40{}h~\char60{}\char60{}~\char40{}id\char62{}\char60{}f\char41{}\char41{}~x~\\
\ttfamily ~~~~~in~cataBTree~g~x\\
\ttfamily ~\\
\ttfamily ~let~balBTree~x~\char61{}~\char40{}p1~\char60{}\char60{}~baldepth\char41{}~x\\
\ttfamily ~\\
\ttfamily ~let~depthBTree~x~\char61{}~\char40{}p2~\char60{}\char60{}~baldepth\char41{}~x\\
\ttfamily ~\\
\ttfamily ~\char47{}\char47{}~\char40{}6\char41{}~Going~polytipic~\char45{}\char45{}\char45{}\char45{}\char45{}\char45{}\char45{}\char45{}\char45{}\char45{}\char45{}\char45{}\char45{}\char45{}\char45{}\char45{}\char45{}\char45{}\char45{}\char45{}\char45{}\char45{}\char45{}\char45{}\char45{}\char45{}\char45{}\char45{}\char45{}\char45{}\char45{}\char45{}\char45{}\char45{}\char45{}\char45{}\char45{}\char45{}\char45{}\char45{}\char45{}\char45{}\char45{}\char45{}\char45{}\char45{}\char45{}\char45{}\char45{}\char45{}\char45{}\char45{}\char45{}\char45{}\char45{}\\
\ttfamily ~\\
\ttfamily ~\char47{}\char47{}~natural~transformation~from~base~functor~to~monoid\\
\ttfamily ~\char47{}\char47{}let~tnat~f~\char61{}\\
\ttfamily ~\char47{}\char47{}~~~~let~theta~\char61{}~uncurry~\char40{}\char60{}\char62{}\char41{}\\
\ttfamily ~\char47{}\char47{}~~~~in~either~\char40{}konst~mempty\char41{}~\char40{}theta~\char60{}\char60{}~\char40{}f~\char62{}\char60{}~theta\char41{}\char41{}\\
\ttfamily ~\char47{}\char47{}\\
\ttfamily ~\char47{}\char47{}~monoid~reduction~\\
\ttfamily ~\char47{}\char47{}\\
\ttfamily ~\char47{}\char47{}let~monBTree~f~\char61{}~cataBTree~\char40{}tnat~f\char41{}\\
\ttfamily ~\\
\ttfamily ~\char47{}\char47{}~alternative~to~\char40{}4\char46{}2\char41{}~serialization~\char45{}\char45{}\char45{}\char45{}\char45{}\char45{}\char45{}\char45{}\char45{}\char45{}\char45{}\char45{}\char45{}\char45{}\char45{}\char45{}\char45{}\char45{}\char45{}\char45{}\char45{}\char45{}\char45{}\char45{}\char45{}\char45{}\char45{}\char45{}\char45{}\char45{}\char45{}\char45{}\char45{}\char45{}\char45{}\char45{}\char45{}\char45{}\char45{}\char45{}\\
\ttfamily ~\\
\ttfamily ~\char47{}\char47{}let~preordt\char39{}~x~\char61{}~monBTree~singl~x\\
\ttfamily ~\\
\ttfamily ~\char47{}\char47{}~alternative~to~\char40{}4\char46{}1\char41{}~counting~\char45{}\char45{}\char45{}\char45{}\char45{}\char45{}\char45{}\char45{}\char45{}\char45{}\char45{}\char45{}\char45{}\char45{}\char45{}\char45{}\char45{}\char45{}\char45{}\char45{}\char45{}\char45{}\char45{}\char45{}\char45{}\char45{}\char45{}\char45{}\char45{}\char45{}\char45{}\char45{}\char45{}\char45{}\char45{}\char45{}\char45{}\char45{}\char45{}\char45{}\char45{}\char45{}\char45{}\char45{}\char45{}\\
\ttfamily ~\\
\ttfamily ~\char47{}\char47{}let~countBTree\char39{}~x~\char61{}~monBTree~\char40{}konst~\char40{}Sum~1\char41{}\char41{}~x\\
\ttfamily ~\\
\ttfamily ~\char47{}\char47{}~\char40{}7\char41{}~Zipper~\char45{}\char45{}\char45{}\char45{}\char45{}\char45{}\char45{}\char45{}\char45{}\char45{}\char45{}\char45{}\char45{}\char45{}\char45{}\char45{}\char45{}\char45{}\char45{}\char45{}\char45{}\char45{}\char45{}\char45{}\char45{}\char45{}\char45{}\char45{}\char45{}\char45{}\char45{}\char45{}\char45{}\char45{}\char45{}\char45{}\char45{}\char45{}\char45{}\char45{}\char45{}\char45{}\char45{}\char45{}\char45{}\char45{}\char45{}\char45{}\char45{}\char45{}\char45{}\char45{}\char45{}\char45{}\char45{}\char45{}\char45{}\char45{}\char45{}\char45{}\char45{}\char45{}\char45{}\char45{}\\
\ttfamily ~\\
\ttfamily ~\char47{}\char47{}type~Deriv\char60{}\char39{}a\char62{}~\char61{}~Dr~of~Bool~\char39{}a~BTree\char60{}\char39{}a\char62{}\\
\ttfamily ~\char47{}\char47{}\\
\ttfamily ~\char47{}\char47{}type~Zipper\char60{}\char39{}a\char62{}~\char61{}~\char91{}~Deriv\char60{}\char39{}a\char62{}~\char93{}\\
\ttfamily ~\char47{}\char47{}\\
\ttfamily ~\char47{}\char47{}let~rec~plug~x~t~\char61{}\\
\ttfamily ~\char47{}\char47{}~~~~match~x~with\\
\ttfamily ~\char47{}\char47{}~~~~\char124{}~\char91{}\char93{}~\char45{}\char62{}~t\\
\ttfamily ~\char47{}\char47{}~~~~\char124{}~\char40{}\char40{}Dr~false~a~l\char41{}\char58{}\char58{}z\char41{}~\char61{}~Node~\char40{}a\char44{}\char40{}plug~z~t\char44{}l\char41{}\char41{}\\
\ttfamily ~\char47{}\char47{}~~~~\char124{}~\char40{}\char40{}Dr~true~a~r\char41{}\char58{}\char58{}z\char41{}~\char61{}~Node~\char40{}a\char44{}\char40{}r\char44{}plug~z~t\char41{}\char41{}\\
\ttfamily ~\char47{}\char47{}\\
\ttfamily ~\char47{}\char47{}\char45{}\char45{}\char45{}\char45{}\char45{}\char45{}\char45{}\char45{}\char45{}\char45{}\char45{}\char45{}\char45{}\char45{}\char45{}\char45{}\char45{}\char45{}\char45{}\char45{}\char45{}\char45{}\char45{}\char45{}\char45{}\char45{}~end~of~library~\char45{}\char45{}\char45{}\char45{}\char45{}\char45{}\char45{}\char45{}\char45{}\char45{}\char45{}\char45{}\char45{}\char45{}\char45{}\char45{}\char45{}\char45{}\char45{}\char45{}\char45{}\char45{}\char45{}\char45{}\char45{}\char45{}\char45{}\char45{}\char45{}\char45{}\char45{}\char45{}\char45{}\char45{}
\end{tabbing}

%----------------- Fim do anexo com soluções dos alunos ------------------------%

%----------------- Índice remissivo (exige makeindex) -------------------------%

\printindex

%----------------- Bibliografia (exige bibtex) --------------------------------%

\bibliographystyle{plain}
\bibliography{cp2021t}

%----------------- Fim do documento -------------------------------------------%
\end{document}
